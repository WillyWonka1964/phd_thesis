\chapter{Conclusions and future work}
\label{conclusions_and_future_work}
\ifpdf
    \graphicspath{{chapter_5/figs/}}
\fi

This thesis has introduced three image processing and analysis pipelines used in the field of endothelial cell biology. In each case the background and acquisition of data has been described. The process of extraction and analysis of data from each dataset has been described. Each technique has been validated, either by comparison to the best available methods or by comparison to a gold standard.

This chapter will conclude this thesis and outline potential directions for future work in the field.

\section{Future Work}
\subsection{High-throughput morphometric analyses of endothelial organelles}
\subsection{Image analysis for platelet dense granule deficiency diagnostics}
The findings from this proof-of-principle HPS study are being prepared for publication, with many possibilities for future studies and improvements to the methodology.

The methodology could be extended by the addition of fluorescent probes to quantify additional platelet structures such as $\alpha$-granules, to provide extra diagnostic information. A further comprehensive survey of platelet granule disorders could also include other markers such as von Willebrands Factor, P-Selectin and serotonin to demarcate other distinct structures within the platelet. The addition of these markers would in turn require additional image processing steps to be implemented. 

This application allows platelets to be fixed and sent to a central facility for diagnosis and re-analysed and imaged at a later date if required. This would relieve pressure on specialised facilities that are dependent on functional analyses that require measurements to be taken within hours of the blood being collected, and also relieve pressure on the patient who is not required to travel to a specialised facility. An automated imaging approach would also reduce any variation in the counting of granules and morphometric data may reveal new disease phenotypes. With enough samples a database of HPS phenotypes could be built and sophisticated analysis could then provide an accurate tool for diagnosis. The extra markers and additional patient data collected could be used to identify HPS phenotypes and used as a more specific diagnostic tool. This is of particular importance as no currently used methods are sensitive or reliable enough to identify milder HPS phenotypes and there is no method currently available that allows alpha and dense granule to be measure in high numbers from the same sample

\subsection{Automated detection and tracking of leukocytes \emph{in vitro}}
The leukocyte detection and tracking workflow has not been extensively tested. A first step in assessing the validity and accuracy of the technique described is to compare the numbers of slow rolling and firmly adhering leukocytes in videos quantified with the detection and tracking routine to those that have been quantified manually. If the methodology described is able to detect similar numbers of these slow rolling or firmly adhering leukocytes it would indicate that this method is able to quantify biological differences in different video sequences.

The sensitivity of the Haar-like features object detection may be improved by adjusting parameters in \emph{detectMultiScale} window, such as the minimum number of neighbours. The object classification works well but has difficulty when particles are very close or overlapping, by allowing more objects in a set area may improve the detection rate.

A number of improvements can also be implemented to improve the particle tracking. An easy improvement would be to remove all leukocyte trajectories that are in the first or last frame of the video, since these cannot be considered as complete trajectories. The number of frames that the tracking algorithm looks ahead could be increased. If a particle is not detected in a frame the trajectory may be preserved.

An alternative method of pruning trajectories, rather than specifying a limit to the minimum number of frames for a trajectory, would be to only accept trajectories as valid if they first appear with a $y$ coordinate near the top of the image and end with a $y$ coordinate near the bottom of the image. This may prove a more valid approach to tracing trajectories.

A final improvement that could be very interesting biologically and allow for much deeper analysis would be to segment the background HUVEC monolayer. A segmentation protocol could be applied to do this or it could even be done manually, as there are usually less than 100 cells to segment in an image. By doing this the role of individual endothelial cells in the leukocyte adhesion cased could be observed and quantified. 

