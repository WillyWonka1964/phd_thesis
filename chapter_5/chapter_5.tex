\chapter{Conclusions and future work}
\label{conclusions_and_future}
\ifpdf
    \graphicspath{{chapter_5/figs/}}
\fi

Scientific endeavour increasingly requires analytical methods combining scientific inquiry, statistical knowledge, and computer programming. Biological insight in cell biology is increasingly reliant on computerised techniques along with advances in microscopy and data acquisition. This body of work has sought to utilise techniques in image processing, data mining, statistics, and machine learning to further advance understanding of cardiovascular processes.

Three methods have been introduced in this thesis to facilitate or enhance the output from established methods in cardiovascular experimentation. The three automated methods described in \autoref{endothelial_morphometry}, \autoref{platelets}, and \autoref{leukocytes} extract quantitative data from microscopy images. Each of these methods has been tested in a biological context, and the performance validated. Two papers have thus far been published using the methods described in this thesis. The methods described in \autoref{endothelial_morphometry} contributed towards Ferraro et al., 2016: `Weibel-Palade body size modulates the adhesive activity of its von Willebrand Factor cargo in cultured endothelial cells'~\cite{Ferraro2016}. The methods described in \autoref{leukocytes} contributed towards Westmoreland et al., 2016, `Super-resolution microscopy as a potential approach to platelet granule disorder diagnosis'~\cite{Westmoreland2016}. 

This chapter contains some concluding remarks on each of the methods introduced in this thesis, with additional statements about their significance and their continued usage. Finally areas for potential improvement and avenues for further work are mentioned.

\subsubsection{High-throughput morphometric analyses of endothelial organelles}
In \autoref{endothelial_morphometry} methods were presented to analyse microscopy images acquired in high-throughput of immunostained endothelial cells and their organelles. Segmentation of endothelial cells and assignment of organelles to these cells enables both whole population level and cellular level analysis. Where the study of endothelial cells and WPBs under different physiological and pathophysiological conditions can be used to gain an understanding of underlying haemostatic processes. Additionally, vWF exocytic pores have been extensively studied with this technique, providing a method that is complementary and potentially more sensitive than the commonly used ELISA method.

The method described here has contributed to several articles in preparation for publication and is frequently used to analyse data sets. Its routine usage should continue to be beneficial to members of the lab. Its usage could in the future be expanded further to include segmentation methods for more endothelial organelles. In addition the semi-automated quality control step could be improved to be fully automated, thus reducing the workload.

\subsubsection{Image analysis for platelet dense-granule deficiency diagnostics}
An automated image analysis pipeline discussed in \autoref{platelets} provides a quantitative, unbiased, rapidly acquired data set, that forms the basis for a platelet dense-granule disorder diagnostic tool. This proof-of-principle study described in \autoref{platelets} has huge potential for further study and development. The effectiveness of this method has been demonstrated in Westmoreland et al., 2016~\cite{Westmoreland2016}.

The described method allows platelets to be fixed and sent to a central facility for diagnosis and re-analysed and imaged at a later date if required. This would relieve pressure on specialised facilities that are dependent on functional analyses that require measurements to be taken within hours of the blood being collected, and also relieve pressure on the patient who is not required to travel to a specialised facility. 

The method could also be extended with the addition of fluorescent markers to label additional platelet structures. These markers could for example include $\alpha$-granules, vWF, P-Selectin, and serotonin. The segmentation and feature extraction of multiply labelled platelets would provide an increasingly rich feature set to improve the diagnostic ability. If more samples were acquired and imaged from healthy volunteers and patients with a multitude of platelet disorders, large data sets could be built up. This would provide fertile ground for more sophisticated and specific diagnostic tools. For example supervised machine learning methods could be used to model subtle disease phenotypes, and then used to classify platelets according to the phenotype represented. This could have the additional benefit of being able to identify milder HPS phenotypes, as there is no method currently available that allows alpha and dense-granule to be measure in high numbers from the same sample.

\subsubsection{Automated detection and tracking of leukocytes \emph{in vitro}}
The technique presented in \autoref{leukocytes} for improving the quantitative yield from leukocyte interactive flow assays has not yet been extensively used, but it represents a promising analytical tool. It has been shown to be highly accurate at detecting and tracking leukocytes over multiple frames. However, it does have some limitations. The object classification has difficulty when particles are very close or overlapping. This problem can be alleviated by reducing the flux of leukocytes and increasing the video length.

An improvement that could be very interesting biologically and allow for much deeper analysis would be to segment the background HUVEC monolayer. A segmentation protocol could be applied to do this or it could even be done manually, as there are usually less than 100 cells to segment in an image. By doing this the role of individual endothelial cells in the leukocyte adhesion cased could be observed and quantified.

\subsection*{Final remarks}
As modern cell biology becomes increasingly a quantitative science it is hoped that the role of bioinformatics gains more recognition as being integral to the process of discovery. Moreover the skillset within the scientific community will need to adapt to better understand and assess the quantification and data analytics methods and their impact on experimental findings. This relatively recent, niche, and interdisciplinary field of bioinformatics will benefit greatly from continued advances in the fields of artificial intelligence and machine learning. These techniques should provide exciting new avenues to further improve image segmentation and recognition tasks.

The methods presented in this thesis were developed in close collaboration with experimentalists at the MRC Laboratory for Molecular Cell Biology in London. Code was concisely written and documented so as to be extensible and easily understandable. It is hoped that these methods will continue to be of use within the laboratory, in this way they can continue to provide unique insight and understanding of endothelial processes. The code base provides a foundation which could be used for the design and architecture of future computational methods. The ideas and explanations presented in this thesis will serve as a useful reference to guide the use and potential development of these tools.
