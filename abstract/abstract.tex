\begin{abstract} % max 300 words
This thesis details the development of computerised image processing and analysis pipelines for quantitative evaluation of microscope image data acquired in endothelial vascular biology experimentation. The overarching objective of this work was to advance our understanding of the cell biology of cardiovascular processes; principally involving haemostasis, thrombosis, and inflammation.

Bioinformatics techniques are increasingly necessary to extract and evaluate information from biological experimentation. In cell biology advances in microscopy and the increased acquisition of large scale digital image data sets have created a need for automated image processing and data analysis. The development, testing and evaluation of three computerised workflows for analysis of microscopy images investigating cardiovascular cell biology are described here.

The first image analysis pipeline extracts morphometric features from high-throughput experiments imaging endothelial cells and organelles. Segmentation of endothelial cells and their organelles followed by extraction of morphometric features provides a rich quantitative data set to investigate haemostatic mechanisms.

A second image processing workflow was applied to platelet images obtained from super-resolution microscopy, and used in a proof-of-principle study of a new platelet dense granule deficiency diagnostic method. The method was able to efficiently differentiate between healthy volunteers and three patients with Hermansky-Pudlak syndrome. This was achieved by segmenting and counting the number of CD63-positive structures per platelet, allowing for the differentiation of patients from control volunteers with 99\% confidence.

The final workflow described is a video analysis that quantifies interactions of leukocytes with an endothelial monolayer. Phase contrast microscopy videos were analysed with a Haar-like features object detection and custom tracking method to quantify the dynamic interaction of rolling leukocytes. This technique provides much more information than a manual evaluation and found to give a tracking accuracy of 92\%.

These three methodologies provide a toolkit to further biological understanding of multiple facets of cardiovascular behaviour.
\end{abstract}
