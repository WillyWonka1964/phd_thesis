\begin{abstract}
This thesis details the development of image analysis pipelines for quantitative evaluation in endothelial vascular biology. The application of advanced image and data analysis techniques to extract quantitative information from images obtained from sophisticated cardiovascular experimentation is helping to build a multi-faceted understanding of endothelial cell biology. 

Of the five studies described, two are immunofluorescent high-throughput studies, two involve flow assays, one of which is recorded as a video, and a final study uses the state of the art superresolution structured illumination microscopy. These diverse image acquisition methods present a variety of challenges for computational evaluation and correspondingly a range of computational tools and approaches have been adopted. Including the use of ImageJ macros, Java development of an ImageJ plugin, Python, OpenCV, R, Bash scripting, PERL, AWK and SED.

This work has been undertaken as a collaborative project split between the Bioinformatics Institute (BII), a member of A*STAR's biomedical sciences institutes in Singapore and the MRC Laboratory for Molecular Cell Biology (LMCB) in London. At BII in Singapore a foundation to image processing and data analysis techniques has been developed including the use of; spatial processing, image segmentation, image inpainting, support vector machines, random forests, the evolving generalised voronoi diagram, decision tree learning, k-means clustering and object recognition.
\end{abstract}
