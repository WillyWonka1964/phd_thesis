\chapter{Introduction}
\label{introduction}
\ifpdf
	\graphicspath{{chapter_1/figs/}}
\fi

\nomenclature[z-HUVEC]{$HUVEC$}{Human Umbilical Vein Endothelial Cell}
\nomenclature[z-WPB]{$WPB$}{Weibel Palade Body}

Increasingly sophisticated and diverse microscopy is being used in cardiovascular experimentation. As a result image analysis approaches are becoming more and more necessary to harvest quantitative data from the often large and complex image based data sets. This thesis describes methods developed for extracting quantitative output from images, which will be useful not only to this research team, but more generally in the field.

The overarching objective of this work was to advance our understanding of the cell biology of cardiovascular processes: haemostasis, thrombosis and inflammation. Where this project specifically aimed to develop a suite of image and video analysis approaches to quantify and aid in the evaluation of data obtained from investigations into endothelial, haemostatic, and inflammatory processes.

In this chapter some background to the research topic and research questions being addressed is given. A general introduction to the various facets of this interdisciplinary research area is also provided, including: an introduction to endothelial cellular biology, the state of the art in microscopy, an overview of techniques in image processing and analysis, and finally a discussion on suitable methods to evaluate and validate image processing pipelines.

\section{Endothelial cellular biology}
\label{introduction:endothelial_cellular_biology}
In complex organisms the circulatory system facilitates the transport of lymph and blood; where the system supporting the circulation of lymph is referred to as the lymphatic system, and the system responsible for the circulation of blood is referred to as the cardiovascular system. The blood circulating in the cardiovascular system is comprised of: plasma, red blood cells, white blood cells, and platelets. In addition to circulating blood the cardiovascular system circulates and transports nutrients, oxygen, carbon dioxide, and hormones. This network is extensive where nearly all cells in the majority of tissue of a vertebrate are located within \SIrange{50}{100}{\micro\meter} of a capillary. This extensive network is crucial to allow blood circulation for the transport of oxygen and nutrients to tissue and the removal of waste material from tissues.

Failure to maintain the physical integrity of the closed, distributed, and pressurised network forming the cardiovascular system leads to bleeding and a loss of pressure. This has a serious impact on the viability of an organism, and thus there is machinery in place for maintaining the physical integrity of the system. Repairing holes in vessel walls is an important aspect of cardiovascular function. A second factor important to cardiovascular system function is the need to allow for controlled exit through the vessel wall where repair of underlying tissue is required. This may be for example at sites of tissue damage, where leukocytes are transported from the circulature into the tissue. Repairing holes in the vessel walls and controlled exit from the vasculature are both supported by the cells lining the vasculature, called endothelial cells.

\subsection{Endothelial cells}
\label{introduction:endothelial_cellular_biology:endothelial_cells}
Endothelial cells line the entire cardiovascular system from the heart to the smallest capillaries. Arteries and veins develop from small vessels composed only of endothelial cells and a basal lamina. Signals from endothelial cells initiate the development of connective tissue cells that form the surrounding layers of the blood-vessel wall~\cite{Alberts2002}.

Endothelial cells form the interface between plasma circulating in the blood and the surrounding tissues, as illustrated in \autoref{figure:introduction:endothelial_mechanisms:blood_vessel}. At this interface position the thin endothelial cell lining assumes the role of an active selective permeability barrier, controlling the movement of materials and white blood cells to and from the bloodstream~\cite{Alberts2002}. 

The endothelial function as a permeable barrier is aided by its organisation into a thin layer frequently only a single cell thick. Additionally endothelial cells have a thin profile, this is demonstrated in \autoref{figure:introduction:endothelial_mechanisms:endothelial_cell}, here, the side profile is much thinner than the plan view. Vascular endothelial cells react to fluid flow by aligning and elongating in the direction of flow~\citeEskin1984}. The endothelial cells respond to physical and chemical stimuli within the circulation to regulate: haemostasis, the vasomotor tone, and the immune and inflammatory responses. Their is also a crucial role played by endothelial cells in the formation of new blood vessels, this process is termed angiogenesis~\cite{Sumpio2002}.

\begin{figure}[htbp]\centering
\begin{minipage}{0.49\textwidth}
	\begin{subfigure}[b]{\linewidth}
		\centering
		\includegraphics[height=17cm]{drawing:blood_vessel}
		\caption{}
		\label{figure:introduction:endothelial_mechanisms:blood_vessel}
		\vspace{1ex}
	\end{subfigure}
\end{minipage}
\begin{minipage}{0.49\textwidth}
	\begin{subfigure}[b]{\linewidth}
		\centering
		\includegraphics[width=\linewidth]{drawing:endothelial_cell}
		\caption{}
		\label{figure:introduction:endothelial_mechanisms:endothelial_cell}
		\vspace{10.5ex}
	\end{subfigure}
	\begin{subfigure}[b]{\linewidth}
		\centering
		\includegraphics[width=\linewidth]{drawing:wpb}
		\caption{}
		\label{figure:introduction:endothelial_mechanisms:wpb}
		\vspace{1ex}
	\end{subfigure}
\end{minipage}
\caption[Endothelial mechanisms for cardiovascular repair]{Diagrams to elucidate the machinery for cardiovascular repair at a vascular, cellular and organelle scale. The diagram in (a) shows the endothelial lining of a blood vessel with circulating platelets and leukocytes, responsible for providing first aid in the vascular environment. Diagram (b) depicts an endothelial cell and its content of the storage organelles Weibel-Palade bodies (WPBs), where one WPB is undergoing exocytosis. Finally, diagram (c) shows the \emph{cigar} like Weibel-Palade body and two proteins: Von Willebrand factor (VWF) essential for the haemostatic response and P-selectin important in the mechanisms of inflammation and the immune response.}
\label{figure:endothelial_mechanisms}
\end{figure}

The endothelial processes in haemostasis and the inflammatory response are largely supplied by the presence of a storage organelle unique to endothelial cells. Weibel-Palade bodies (WPBs) provide a rapid-response initiation of haemostasis and inflammatory responses within the dynamic and rapidly changing vascular environment.

\subsection{Weibel-Palade bodies}
\label{introduction:endothelial_cellular_biology:wpb}
Endothelial cells contain the storage organelles Weibel-Palade bodies (WPBs), which act as a mediator for vital physiological processes (see \autoref{figure:introduction:endothelial_mechanisms:endothelial_cell} and \autoref{figure:introduction:endothelial_mechanisms:wpb}). The endothelial specific rapid-response WPB organelles measure between \SIrange{0.1}{0.3}{\micro\meter} in diameter and \SIrange{0.5}{5.0}{\micro\meter} in length~\cite{Ferraro2014} and act as a mediator for inflammation and haemostasis.

The WPB was first discovered in 1964 by Ewald Weibel and George Palade, it was described as a `rod-shaped cytoplasmic component, which consists of a bundle of fine tubules, enveloped by a tightly fitted membrane'~\cite{Weibel1964}. The secretory vesicles WPBs contain a set of molecular cargos that can be released in response to endothelial activation from an external stimulus. Activation initiates an exocytic response to mediate endothelial processes via cellular signalling or mechanical stress. An illustration of a WPB undergoing exocytosis is displayed in \autoref{figure:introduction:endothelial_mechanisms:wpb}. Two key molecules released in WPB exocytosis are von Willebrand factor (vWF) and P-selectin, which are critical for haemostasis and inflammation, respectively.

The central role of WPBs in haemostasis is provided by its content of the platelet-recruiting cargo protein vWF~\cite{Wagner1982}. The multimeric vWF protein underpins primary haemostasis as well as acting to chaperone the blood clotting protein factor VIII. Tubules of vWF are stored within WPBs, giving WPBs a distinctive `cigar-shape', as illustrated in the diagram in \autoref{figure:introduction:endothelial_mechanisms:wpb}.

It was discovered in 1989 that WPB also contain in their membrane, the leukocyte adhesion receptor P-selectin~\cite{Bonfanti1989,McEver1989}. This discovery revealed a second role of WPBs for the recruitment of leukocytes to the site of injury during inflammation. The combination of the haemostatic response provided by vWF and the inflammatory response provided by P-selectin makes WPBs in Denisa Wagner's words `the perfect first aid kit after an insult to the vasculature'~\cite{Weibel2012}.

\subsubsection{Von Willebrand Factor}
\label{introduction:endothelial_cellular_biology:vwf}
The large multimeric, multi-domain glycoprotein vWF is expressed by endothelial cells and platelets. VWF is involved in the initiation of primary and secondary haemostasis~\cite{Metcalf2008}. In primary haemostasis where coagulation has not yet developed, vWF acts to form a loose platelet plug to control bleeding. Following this, secondary haemostasis occurs through an enzymatic coagulation cascade resulting in the formation of cross linked fibrin monomers, stabilising the loose platelet plug to form a clot.

VWF is a cofactor to Factor VIII, preventing its proteolysis within the blood stream~\cite{Sadler1998}. Von Willebrand's disease (vWD) is a condition arising from quantitative or qualitative abnormalities in vWF. Four hereditary types of vWD exist, the current classification of vWD distinguishes disorders arising from partial type I or complete type III deficiencies and from qualitative defects type II and platelet-vWD or pseudo-vWD. Type II vWD is subdivided into into subtypes A, B, N, and M, each with distinct classes of functional abnormalities~\cite{Ewenstein1997}. Platelet-type vWD or pseudo-vWD is an autosomal dominant genetic defect of the platelets. The vWF is qualitatively normal and genetic testing of the von Willebrand gene and vWF protein reveals no mutational alteration. The defect lies in the qualitatively altered glycoprotein1 (GP1) receptor on the platelet membrane which increases its affinity to bind to the von Willebrand factor.

\subsubsection{P-selectin}
\label{introduction:endothelial_cellular_biology:p-selectin}
Additional components of WPBs have been identified giving further indication as to the organelle's cellular function. Of particular importance are P-selectin~\cite{Bonfanti1989,McEver1989}, Rab27a~\cite{Hannah2003} and CD63~\cite{Vischer1993}. P-selectin is a leukocyte receptor and plays an essential role in the initial recruitment of leukocytes to the site of injury during inflammation. P-selectin is found on the apical surface of endothelial cells and recruits leukocytes when it is exposed to blood via WPB exocytosis, binding the glycoprotein ligand (PSGL)-1 onto the leukocyte surface. The suggested function of Rab27a is to prevent exocytosis of immature WPBs by anchoring them to the actin cytoskeleton~\cite{Nightingale2009}. The tetraspanin CD63 is known to be a component of WPBs, however no endothelial specific function has yet been assigned to this protein. The synthesis and storage of the various WPB protein components is dynamic and can alter rapidly under different cellular conditions.

\subsection{Haemostasis}
Hemostasis is the process by which a barrier to blood loss is created at the site of blood vessel injury. Hemostasis most commonly occurs in the smallest veins and arteries, the venules and arterioles, as well as in capillaries. 

Primary hemostasis is the first phase in the hemostatic process. Here, platelets interact among themselves and with the injured blood vessel to create the primary hemostatic plug. This plug temporarily arrests bleeding. However, it is fragile and can be dislodged with ease from the vessel wall.

Secondary Hemostasis
During secondary hemostasis, the primary hemostatic plug is strengthened and stabilized by the deposition of insoluble strands of fibrin. Fibrin is produced through a series of complex biochemical reactions when soluble plasma proteins, known as coagulation factors, associate with the platelet plug and the injured vessel. The addition of fibrin to the primary hemostatic plug creates the secondary hemostatic plug.

When the vessel injury has healed, the secondary hemostatic plug, no longer needed, is broken down and removed by additional components of the hemostatic system in a process known as fibrinolysis.

Hemostasis is initiated by damaged blood vessels. One early response to injury is to constrict or narrow the lumen of the arterioles, thereby minimizing both the flow of blood to the wounded area and the loss of blood from the wound. This initial vasoconstriction occurs and is balanced by a complementary process of vasodilation. 

Hemostasis inhibits clot formation in the absence of injury by maintaining a nonreactive environment for the components of the hemostatic system. Endothelial cells are responsible for modulating functions that both form and prevent blood clots. In the absence of vessel injury, the negatively charged surface of the endothelial cell repels negatively charged circulating proteins and platelets. However, when injury to the endothelium occurs, and under the high shear flow conditions that occur at the site of vessel injury, platelets come in contact with subendothelial collagen and vWF, causing platelet adhesion and initiating the process of primary hemostasis. 

One of the primary thrombogenic functions of the endothelium is therefore production and processing of vWF. During vessel injury, vWF is secreted into the subendothelial tissue and into the plasma from the luminal side of the endothelium. vWF plays an important role in the initial stage of clot formation by binding to collagen fibers in the extracellular matrix and supporting the binding of platelets, as well as recruiting platelets through long-range contacts extending from the endothelial surface into the plasma.




\subsection{Inflammation}

A variety of adhesion receptors have been implicated in leukocyte endothelial interactions and are thought to play roles in recruitment, arrest, and extravasation of leukocytes. A critical role is played by the family of $\beta$2 or leukocyte integrins~\cite{Springer1990, Mayadas1993}. The genetic disease leukocyte adhesion deficiency I (LADI) is caused by a mutation of the $\beta$2 integrin gene. The increased concentration of circulating leukocytes in patients with LADI implies that $\beta$2 or leukocyte integrins play an important role in leukocyte extravasation. Patients with LADI are susceptible to frequent bacterial and fungal infections~\cite{Anderson1987}. Effective arrest of leukocytes at an inflammation site requires local activation of circulating leukocytes. The circulating leukocytes transiently interact with the activated endothelium, reducing their velocity in the direction of flow, allowing activation of their $\beta$2 integrins, leading to arrest and extravasation. This transient interaction is manifested as rolling of the leukocytes along the vessel wall in areas of inflammation~\cite{Mayadas1993, Atherton1972}.

The selectins are a family of adhesion receptors that have been shown to mediate rolling of leukocytes. There are three known selectins; L-Selectin, E-Selectin and P-Selectin, which are encoded by three closely linked genes~\cite{Watson1990}. The integral membrane protein leukocyte receptor P-selectin is stored within the membrane of Weibel-Palade bodies (WPBs) of endothelial cells~\cite{Bonfanti1989, McEver1989}, from where it is delivered to the cell surface within minutes after secretagogue-triggered exocytosis~\cite{McEver2002}. P-selectin plays a key early role in the inflammatory trafficking of leukocytes, being the first receptor involved in recruiting leukocytes from flowing plasma to the endothelial surface~\cite{Larsen1989}.


\section{Microscopy}


Visual inspection of cells under a microscope is a basic yet fundamental technique in cell biology. Traditionally, experimental data from microscopy would be measured and assessed by a trained biologist. The human visual system is adept at intuitively interpreting visual information, and endowed with the ability to filter irrelevant variations in illumination and contrast. Recent advances in optical technologies and instrumentation are providing previously unimagined capabilities. Robotic microscopy and modern immunostaining techniques have increased the speed and capacity to gather microscopic image data from experiments. This has served to amplify the utility and necessity of visual inspection and measurement of cells. Expert manual inspection and annotation is not feasible for detection of subtle cell population differences over tens or hundreds of thousands of cells. Visual analysis has therefore become a bottleneck in performing large image-based high-throughput screens. To address this, advances in bioinformatics and high-throughput computational automated analysis tools have been developed to extract quantitative information from microscopy images.

Automated image analysis does provide some substantial benefits over manual scoring, including: providing unbiased highly quantitative scoring, highly reproducible, able to detect subtle population differences, is fast, eliminates tedious manual labour and is able to simultaneously collect multiple features~\cite{Jones2006}. 

\subsection{Resolution}
The resolving power or resolution limit of the optical system is defined as the minimum distance apart of two objects in order for them to be resolved as separate objects. This distance is equal to the smallest point source in the image, and is given by Abbe's equation~\cite{Abbe1873},
\begin{equation}
r=0.61\frac{\lambda}{NA},
\label{equation:abbe}
\end{equation}
where, $r$ is the resolution limit, $\lambda$ is the wavelength and $NA$ the numerical aperture of the optical system.


\subsection{Immunofluorescence}
\subsection{Phase contrast}

\subsection{Error sources}
Quantitative measurements in fluorescence microscopy contain some amount of error, this may be introduced by the specimen, the microscope or the detector. To minimise sources of error, in high-throughput morphometry of endothelial organelles requires consideration of, both the image processing and experimental design. The imaging system should be configured for optimal signal detection, low background and low noise. Acquired images should be in focus over the full dynamic range of the camera whilst avoiding pixel saturation. To maintain pixel values digital files must be stored in original raw format or using lossless compression.


\section{Image Processing}
A general approach to experimentation in life sciences is to investigate defined research questions or hypotheses, via quantitative analyses of experimental data. A single experiment or a series of experiments may be constructed to investigate a hypothesis. For a hypothesis to be accepted, results must be reproducible over multiple experiments and corroborated by alternative experimental methods. This approach of experimentation, interpretation and re-experimentation is effective when applied to high-throughput morphometric analyses of endothelial organelles.

The experimental loop displayed in Figure~\ref{figure:endothelial_morphometry:introduction:experimental_workflow} shows typical stages in experiment design and analysis. The top row of processes deals with the biological hypothesis and experimental setup, whilst the bottom row of processes are relating to data analysis and interpretation of the results. The first step in the loop is specification of a hypothesis, followed by: defining experimental groups to test the hypothesis, assignment of samples to those groups and preparation of the specimen. The bottom row of processes converts the raw data into a form that can quantitatively address the initial hypothesis. Collection of data, aggregation of data and additional analysis allows for interpretation of the data. Typically, investigation of a hypothesis leads to investigation of further hypotheses and the loop is self-perpetuating.

\begin{figure}[htbp!]
	\centering
	\includegraphics[width=1.0\textwidth]{experimental_workflow}
	\caption[Experimental workflow in life science imaging]{A general experimental workflow in life science imaging. Adapted from Prodanov and Verstreken, 2012~\cite{Prodanov2012}}
	\label{figure:endothelial_morphometry:introduction:experimental_workflow}
\end{figure}

At each step in the processing workflow in Figure~\ref{figure:endothelial_morphometry:introduction:experimental_workflow} the volume of output data is decreased, with a corresponding increase in the complexity of generated information or derived data. For example, acquired images of cells can be transformed into a set of morphometric and intensity features. Each feature has a different semantic context, such as: nucleus area or cell perimeter. In the acquired raster images of cells the biological object is only implicitly present, but in the derived data the biological object is explicitly constructed. The explicit information contained within the raster image for example about staining distribution and illumination is lost. The process of object segmentation and feature extraction is therefore accompanied by irreversible reduction of the input information. At each step in the workflow the information in the previous step is transformed into contextual data, called metadata. The information complexity increase is thereby also mapped to an increase of the complexity of the data structure~\cite{Prodanov2012}.

For example the extraction of morphometric and pixel intensity measurements leads to a reduced representations of the image features of interest. These features have higher information complexity compared to the raw data, but there is an irreversible information loss by the process of measurement. To replicate the measurements from the original data, the same image processing algorithms must be applied. Therefore, measurements are only implicitly present in images~\cite{Prodanov2012}.

\subsection{Morphological image processing}
In a biological context morphology refers to the study of form and structure of an organism. The mathematical context of morphology is as a tool for extracting image components that are of use in representation of region shape. In morphological image processing the inputs are images but the outputs are attributes extracted from those images.

Morphometric analyses of endothelial organelles sets out to obtain features relating to the shape and pixel intensity of organelles. A number of steps were performed in order to extract morphological features, from endothelial cell images. Typically, images may be preprocessed to reduce noise and enhance the contrast between certain features. Different segmentation approaches will be attempted and parameters optimised to subdivide the image into its constituent regions or objects. A binary mask of the relevant segmented objects can then be used to measure properties of the labeled image regions. Properties include morphometry and with reference to the original image pixel intensity information. The resulting measurements can be combined and evaluated statistically to infer biological meaning.





\subsection{Image enhancement}
\subsection{Segmentation}
\subsection{Morphological processing}
\subsection{Object detection}

\section{Machine Learning}
\subsection{SVM}
\subsection{Random Forest}
\subsection{CART}

\section{Validation}
The evaluation of image segmentation is a necessary and often overlooked stage of image analysis.
\subsection{Dice and Jaccard}
\subsection{Hungarian algorithm}
\subsection{Training and Validation}

\section{Overview}
\label{introduction:background}
An initial aim of this project was to further develop an image processing tool that has been used to quantify the morphometric features of a large population of the endothelial organelles Weibel-Palade bodies (WPBs). This computational tool, discussed in \autoref{endothelial_morphometry}, can be applied to large scale morphometric studies of WPBs under different physiological and pathophysiological conditions to gain an understanding of underlying haemostatic processes. Data mining of WPB morphometric features at a cellular level and whole population level could then be employed along with statistical, machine learning and visualisation techniques to gain further biological insight.

An adaptation of the WPB morphometric computational analysis tool, discussed in \autoref{endothelial_morphometry:image_processing:exit_sites}, has been applied to a high-throughput study of the morphometry of WPB exocytic exit sites. A WPB exocytic exit site is observed on the endothelial plasma membrane as a fusion pore, by immunofluorescent labelling of endothelial cells with vWF internally and externally. Through counting absolute numbers of exit sites, the number of exit sites per cell and the areas of exit sites, this analysis can be used to garner information about von Willebrand Factor (vWF) release that is complementary to and potentially more sensitive than the commonly used enzyme-linked immunosorbent assay (ELISA) method. This study is also proving to be useful in understanding the cellular mechanisms that are employed by endothelial cells in controlling vWF release.

The release of vWF into the blood stream occurs when WPBs fuse with the plasma membrane, the vWF unfurls to form long platelet-capturing strings. These platelet-decorated strings perform a vital role in primary haemostasis. An unbiased image analysis workflow has been developed to segment and quantify flourescent stained images of nuclei, vWF strings and platelets under flow from human umbilical vein endothelial cells (HUVECs). A description of this automated image processing workflow is given in section. Currently, published analyses of strings have focused on their number, and how this changes in different physiological conditions. Our new workflow will capture both number and length and we have begun to uncover how this also changes in response to endothelial physiology. 

The vWF strings play an important role in capture of platelets which contribute to haemostasis. Within platelets are granules containing bioactive pro-haemostatic molecules.  Genetic mutations can reduce the number or capacity of platelet granules affecting their functional ability. Genetic abnormalities such as in Hermansky-Pudlak syndrome (HPS) lead to abnormalities in numbers or capacity of platelet granules resulting in bleeding disorders. Diagnostic imaging of platelets to count their dense granules has traditionally required electron microscopy, due to the small size of the platelet granules. Dense granules are approximately \SI{150}{\nano\metre} wide and light microscopy has historically been limited to resolutions of  \SIrange{200}{300}{\nano\metre} depending on the technique. However, developments in super-resolution microscopy now allow sufficient resolution in light microscopy to effectively image platelet granules. An automated image analysis pipeline (section~\autoref{platelets}) that provides a quantitative, unbiased, rapidly acquired dataset forms the basis for a platelet dense granule disorder diagnostic tool. Using the antigen CD63 as a marker, granules are segmented and assigned to their respective platelets. A proof-of-principle dataset with seven healthy controls and three patients with platelet granule abnormalities caused by HPS, has shown the effectiveness of this technique, and this first dataset is being prepared for publication.

Finally, a workflow is being developed (\autoref{leukocytes}) for video analysis of leukocyte extravasation. This is part of the innate immune response whereby stages in the leukocyte adhesion cascade from: capture, rolling, slow rolling and firm adhesion, can be imaged under flow in vitro as a time series. Presently, analysis from such videos involves, simply manually counting leukocytes that are slow rolling or firmly adhering. It is proposed that by accurately detecting the leukocytes in each frame and linking them between frames, information about the trajectories of each leukocyte over an endothelial cell surface could be acquired. This analytical technique will allow for a more thorough analysis of the leukocyte adhesion cascade than a manual counting approach.

The development of this suite of image processing tools will help to build a multifaceted picture of the endothelial processes of haemostasis, thrombosis and inflammation. The image processing tools, together with the analysis of their output, allows for the extraction of a great deal of quantitative information to underpin significant advances in the mechanisms that are at the heart of haemostasis and of the initiation of inflammation.

