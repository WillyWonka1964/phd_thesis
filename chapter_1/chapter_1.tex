\chapter{Introduction}
\label{introduction}
\ifpdf
	\graphicspath{{chapter_1/figs/}}
\fi

\nomenclature[z-HUVEC]{$HUVEC$}{human umbilical vein endothelial cell}
\nomenclature[z-WPB]{$WPB$}{Weibel Palade body}
\nomenclature[z-TGN]{$TGN$}{trans-Golgi network}
\nomenclature[z-vWF]{$vWF$}{von Willebrand factor}
\nomenclature[z-vWD]{$vWD$}{von Willebrand disease}
\nomenclature[z-DIC]{$DIC$}{differential interference contrast}
\nomenclature[z-SIM]{$SIM$}{structured illumination microscopy}
\nomenclature[z-AI]{$AI$}{artificial intelligence}
\nomenclature[z-CCD]{$CCD$}{charged-coupled device}
\nomenclature[z-CARS]{$CARS$}{coherent anti-Stokes Raman spectroscopy}
\nomenclature[z-SHG]{$SHG$}{second-harmonic generation microscopy}
\nomenclature[z-ELISA]{$ELISA$}{enzyme-linked immunosorbent assay}
\nomenclature[z-SNR]{$SNR$}{signal-to-noise ratio}

Modern biology is an increasingly quantitative science in which higher level insights are being obtained by delving deeper into the intricate mechanisms of biological processes. An abstract approach using statistics and modelling is required to infer patterns and trends that cannot be recognised intuitively. Increasingly sophisticated and diverse microscopy techniques are yielding larger and more complex data sets. Optical technologies and instrumentation are providing previously unimagined capabilities in image acquisition. For instance robotic microscopy and modern immunostaining methods have increased the speed and capacity by which microscopes gather image data from experiments. The consequences of these trends has led to the necessity of automated computerised image analysis approaches to harvest data from the image based data sets. These analyses have been facilitated by the rapid advances in computer hardware and improved analytical techniques.

This thesis describes computerised methods developed for extracting and interpreting quantitative output from images obtained from experiments investigating endothelial, haemostatic, and inflammatory processes. The overarching objective of this work was to advance our understanding of the cell biology of cardiovascular processes, principally involving haemostasis, thrombosis, and inflammation. In this chapter a general introduction to the various facets of this interdisciplinary research is provided. Starting with an introduction to endothelial cellular biology in \autoref{introduction:endothelial_cellular_biology}, followed by a description of microscopy techniques in \autoref{introduction:microscopy}, and finally an introduction to digital imaging, and techniques in image processing and analysis is described in \autoref{introduction:image_processing}.

\section{Endothelial cellular biology}
\label{introduction:endothelial_cellular_biology} 
In complex organisms the circulatory system facilitates the transport of lymph and blood; the system supporting the circulation of lymph is referred to as the lymphatic system, and the system responsible for the circulation of blood is referred to as the cardiovascular system. The cardiovascular system is a closed network comprised of: blood, the heart, and blood vessels. It is extensive, such that nearly all cells in the majority of tissues of a vertebrate are located within \SIrange{50}{100}{\micro\meter} of a capillary~\cite{Alberts2002}. The blood circulating in the cardiovascular system is comprised of: plasma, red blood cells, white blood cells, and platelets. In addition to circulating blood the cardiovascular system circulates and transports nutrients, oxygen, carbon dioxide, and hormones. This extensive transport network is crucial to allow blood circulation for the movement of oxygen and nutrients to tissue, and the removal of waste material from tissues.

Failure to maintain the physical integrity of the closed, distributed, and pressurised network forming the cardiovascular system leads to bleeding and a loss of pressure. These have serious impact on the viability of an organism, and thus there are mechanisms in place for maintaining the physical integrity of the system. Two principle mechanisms that maintain the physical integrity of the cardiovascular system and underpin the work for this thesis are:
\begin{itemize}
    \item the system's ability to repair holes in vessel walls, and
    \item the ability to allow for controlled exit through the vessel wall where repair of underlying tissue is required.
\end{itemize}
Controlled exit from the vasculature may be for example at sites of tissue damage, where leukocytes are transported from the circulature into the tissue. Repairing holes in the vessel walls and controlled exit from the vasculature are both actively supported by those cells lining the vasculature, called endothelial cells.

\subsection{Endothelial cells}
\label{introduction:endothelial_cellular_biology:endothelial_cells}
Endothelial cells line the entire cardiovascular system from the heart to the smallest capillaries. They are fundamental to the formation of the vascular network, including the formation of capillaries, arteries, and veins. Development of connective tissue surrounding a blood vessel wall is initiated by signals from endothelial cells~\cite{Alberts2002}. In this way arteries and veins can develop from small vessels composed only of endothelial cells and a basal lamina.

As a lining to the cardiovascular system, endothelial cells form the interface between plasma circulating in the blood and the surrounding tissues as illustrated in \autoref{figure:introduction:endothelial_mechanisms:blood_vessel}. At this interface position the thin endothelial cell lining assumes the role of an active selective permeability barrier, controlling the movement of materials and white blood cells to and from the bloodstream~\cite{Alberts2002}. The endothelial function as a permeable barrier is aided by its organisation into a thin layer only a single cell thick. Cells in the endothelial monolayer also tend to be flat with a thin profile as can be seen in \autoref{figure:introduction:endothelial_mechanisms:endothelial_cell}.

\begin{figure}[htbp]\centering
\begin{minipage}{0.49\textwidth}
	\begin{subfigure}[b]{\linewidth}
		\centering
		\includegraphics[height=17cm]{drawing:blood_vessel}
		\caption{}
		\label{figure:introduction:endothelial_mechanisms:blood_vessel}
		\vspace{1ex}
	\end{subfigure}
\end{minipage}
\begin{minipage}{0.49\textwidth}
	\begin{subfigure}[b]{\linewidth}
		\centering
		\includegraphics[width=\linewidth]{drawing:endothelial_cell}
		\caption{}
		\label{figure:introduction:endothelial_mechanisms:endothelial_cell}
		\vspace{10.5ex}
	\end{subfigure}
	\begin{subfigure}[b]{\linewidth}
		\centering
		\includegraphics[width=\linewidth]{drawing:wpb}
		\caption{}
		\label{figure:introduction:endothelial_mechanisms:wpb}
		\vspace{1ex}
	\end{subfigure}
\end{minipage}
\caption[Endothelial mechanisms for cardiovascular repair at a vascular, cellular and organelle scale]{Diagrams to elucidate the machinery for cardiovascular repair at a vascular, cellular and organelle scale. The diagram in (a) shows the endothelial lining of a blood vessel with circulating platelets and leukocytes, responsible for providing \emph{first aid} in the vascular environment. In this diagram is an example of von Willebrand Factor (vWF) recruitment of platelets at the site of vascular injury and leukocyte extravasation.  Diagram (b) depicts an endothelial cell and its content of the storage organelles Weibel-Palade bodies, where one Weibel Palade body is undergoing exocytosis. Finally, diagram (c) shows the \emph{cigar-shape} WPB and two proteins: von Willebrand factor essential for the haemostatic response and P-selectin important in the mechanisms of inflammation and the immune response.}
\label{figure:endothelial_mechanisms}
\end{figure}
Vascular endothelial cells must respond rapidly to the changing vascular environment. They respond to fluid flow by aligning and elongating in the direction of flow~\cite{Eskin1984}. Endothelial cells also respond to physical and chemical stimuli within the circulation to regulate: haemostasis, the vasomotor tone, and the immune and inflammatory responses. Endothelial cells play a crucial role in angiogenesis, the formation of new blood vessels~\cite{Sumpio2002}.

The endothelial contribution to processes in haemostasis and the inflammatory response are largely supported by the presence of a unique storage organelle. These organelles, called Weibel-Palade bodies (WPBs) provide a rapid-response initiation of haemostasis and inflammatory responses within the dynamic and rapidly changing vascular environment.

\subsection{Weibel-Palade bodies}
\label{introduction:endothelial_cellular_biology:wpb}
The WPB was first discovered in 1964 by Ewald Weibel and George Palade, it was described as a `rod-shaped cytoplasmic component, which consists of a bundle of fine tubules, enveloped by a tightly fitted membrane'~\cite{Weibel1964}. They were later shown to be secretory vesicles and to contain a set of molecular cargoes that can be released in response to endothelial activation from an external stimulus. Activation initiates an exocytic response to mediate endothelial processes via cellular signalling or mechanical stress. WPB organelles measure between \SIrange{0.1}{0.3}{\micro\meter} in diameter and \SIrange{0.5}{5.0}{\micro\meter} in length~\cite{Ferraro2014}. They carry molecular cargoes that can be secreted to act as  mediators for inflammation and haemostasis.

Two key molecules released in WPB exocytosis are von Willebrand factor (vWF) and P-selectin, which are critical to haemostasis and inflammation, respectively. Tubules of vWF are stored within WPBs, giving WPBs a distinctive `cigar shape', as illustrated in the diagram in \autoref{figure:introduction:endothelial_mechanisms:wpb}. The central role of WPBs in haemostasis is provided by its content of the platelet-recruiting cargo protein vWF~\cite{Wagner1982}. Strands of vWF unfurl to capture circulating platelets and form a platelet plug, these strands of vWF can be seen in the WPB undergoing exocytosis in \autoref{figure:introduction:endothelial_mechanisms:wpb}. This mechanism is discussed further in \autoref{introduction:endothelial_cellular_biology:vwf_in_haemostasis}. As well as underpinning primary haemostasis the multimeric vWF protein acts to chaperone the blood clotting protein factor VIII.

It was discovered in 1989 that WPBs also contain the leukocyte adhesion receptor P-selectin in their limiting membrane~\cite{Bonfanti1989,McEver1989}. This discovery revealed a second role of WPBs in the recruitment of leukocytes to the site of injury to support an inflammatory response (see \autoref{introduction:endothelial_cellular_biology:p-selectin_inflammatory_cascade}). The combination of the haemostatic response provided by vWF and the inflammatory response provided by P-selectin makes WPBs in Denisa Wagner's words `the perfect first aid kit after an insult to the vasculature'~\cite{Weibel2012}.

\subsection{The role of Von Willebrand Factor in haemostasis}
\label{introduction:endothelial_cellular_biology:vwf_in_haemostasis}
The physiological response at the site of blood vessel injury that reduces and stops bleeding by creating a barrier to blood loss is called haemostasis. A haemostatic equilibrium is integral to the circulatory system, where an imbalance can lead to morbidity and mortality~\cite{Rasche2001}. Secretion of vWF from endothelial cells initiates haemostasis after injury and provides several essential functions:
\begin{itemize}
\item to bind and transport blood clotting protein factor VIII,
\item to mediate platelet adhesion to reactive surfaces, and
\item to mediate platelet aggregation and thrombus growth~\cite{Ruggeri2007}.
\end{itemize}


\subsubsection{Haemostasis}
The process of haemostasis involves multiple contributions from the vascular system, the coagulation system, the fibrinolytic system, from platelets, the kinin system, from serine protease inhibitors, and from the complement system~\cite{Boon1993}. This complex interconnected process is orchestrated via a sequence of three events:
\begin{enumerate}
\item vascular constriction,
\item the formation of a platelet plug, and
\item the formation of a blood clot as a result of blood coagulation~\cite{Hall2011}.
\end{enumerate}

The first event triggered following vessel injury is vascular constriction or vasoconstriction. This is a localised response by the smooth muscle surrounding the endothelium that causes the vessel to narrow, reducing the diameter of the lumen and the flow of blood to the wounded area, thereby reducing the loss of blood via the wound. To prevent further extensive blood loss additional haemostatic mechanisms are initiated leading to the formation of a blood clot.

Within seconds after vascular injury platelets interact with each other and with the injured vessel. They start to adhere to and accumulate at the site of the damage to form a primary haemostatic plug. Endothelial cells surrounding the vessel injury initiate the capture of platelets by releasing their vWF molecular cargo. In order to recruit circulating platelets strands of vWF are anchored to the plasma membrane of the endothelial cell, and to exposed subendothelial matrix. The accumulated platelet plug acts to temporarily arrest bleeding, it can however easily be broken or dislodged from the vessel wall. Therefore a secondary haemostatic response is required~\cite{Rasche2001}.

Secondary haemostasis is the third part of the haemostatic mechanism, involving the strengthening and stabilising of the primary haemostatic plug. This is achieved by the addition of insoluble strands of fibrin, occurring at approximately \SIrange{15}{20}{\second} after severe injury or \SIrange{1}{2}{\minute} after minor injuries~\cite{Hall2011}. The fibrin is produced through a series of biochemical reactions involving coagulation factors associating with the platelet plug and injured vessel. The addition of fibrin to the primary haemostatic plug creates the secondary haemostatic clot~\cite{Pocock2013}. The secondary haemostatic platelet clot stabilised by strands of fibrin can be seen in \autoref{figure:introduction:endothelial_mechanisms:blood_vessel}. This secure secondary haemostatic clot remains until the vessel injury has healed, upon which it is degraded and removed by further components of the haemostatic system, in a process known as fibrinolysis.

The cardiovascular system inhibits clot formation in the absence of injury by maintaining a nonreactive environment for the components of the haemostatic system. This is achieved by modulating functions that form and prevent blood clots. In the absence of vessel injury, electrostatic repulsion exists between the surface of the endothelial cell and proteins or platelets. Following an endothelial injury however, high shear flow conditions occur, due to perturbed flow, endothelial cells are activated, and platelets come in contact with subendothelial collagen and vWF. This causes platelet adhesion and initiates the process of primary haemostasis~\cite{Sumpio2002}.

The production and processing of vWF is therefore one of the primary thrombogenic functions of endothelial cells. During vessel injury, vWF is secreted into the plasma from the luminal side of the endothelium. VWF plays an important role in the initial stage of clot formation by binding to collagen fibers in the extracellular matrix and supporting the binding of platelets, as well as recruiting platelets through long-range contacts extending from the endothelial surface into the plasma~\cite{DeCeunynck2013}.

\subsubsection{Von Willebrand Factor}
The large, multimeric, multi-domain glycoprotein vWF is expressed by endothelial cells and platelets. In endothelial cells the vWF is produced as ultra-large vWF in WPBs, in platelets vWF is present in $\alpha$-granules, and it is also present in  subendothelial connective tissue following basolateral constitutive secretion from the adjacent endotheium~\cite{Sadler1998}. It has been established that synthesis of vWF is a prerequisite for WPB formation, this is demonstrated by the absence of WPBs in endothelial cells from vWF deficient animals~\cite{Serpe2008}. In the bloodstream vWF is a cofactor to factor VIII, preventing its proteolysis when vWF is depleted factor VIII is rapidly removed from the blood stream~\cite{Sadler1998}.

In endothelial cells vWF is synthesised at the endoplasmic reticulum as a monomer known as \emph{pre-pro-VWF}. Further processing along the secretory pathway occurs following its arrangement into dimers in the endoplasmic reticulum. At the trans-Golgi network (TGN) the vWF is folded into helical tubules~\cite{Ewenstein1987}, that then co-associate to drive the formation of organelles recognisable as WPBs. At this time, cleavage of the pro-peptide and the formation of vWF concatamers of up to 50 dimers long begins to occur, and continues from early formation to full maturity.

\subsection{The role of P-selectin in the inflammatory response}
\label{introduction:endothelial_cellular_biology:p-selectin_inflammatory_cascade}
A number of physiological changes occur following tissue injury, where tissue injury may occur from traumatic injury, infection or cellular necrosis for example. These physiological changes can include local vasodilation, increased permeability of the capillaries to plasma proteins, and infiltration of the damaged tissues by leukocytes~\cite{Hall2011}. This set of changes constitutes the inflammatory response, in which the primary objective is to localise and eradicate the irritant and repair the surrounding tissue. The infiltration of leukocytes to damaged tissue is facilitated largely by the endothelial cells in a process known as extravasation, which is begun via the leukocyte adhesion cascade. Specifically the leukocyte receptor P-selectin present on WPBs initiates the primary step of leukocyte capture in the inflammatory response.

\subsubsection{Inflammatory response}
Leukocytes migrate from circulating in the blood to the surrounding tissue at the site of tissue damage or inflammation. This transfer of leukocytes out of the vasculature is known as extravasation and forms part of the innate immune response. The arrest of circulating leukocytes and their exit from the vasculature relies on appropriate expression and coordination of cell surface adhesion receptors on the leukocytes and endothelium~\cite{Mayadas1993}. The series of adhesion and activation events from the arrest of leukocytes leading to their extravasation is known as the leukocyte adhesion cascade and is shown in \autoref{figure:leukocyte_adhesion_cascade}. A sequence of five steps have been identified in this process. From the capture of the leukocyte, to rolling, slow rolling, adhesion strengthening and paracellular or transcellular migration. The leukocytes are then able to migrate through the basement membrane to the site of damage~\cite{Ley2007}.

\begin{figure}[htbp!]
	\centering
	\includegraphics[width=1.0\textwidth]{drawing:leukocyte_adhesion_cascade}
	\caption[The leukocyte adhesion cascade]{An illustration of the key stages in the leukocyte adhesion cascade, along with an approximate timeline for the involvement of selectins and integrins in leukocyte adhesion. Transmigration maybe paracellular or transcellular, here the paracellular case is shown. Adapted from Ley et al., 2007~\cite{Ley2007}} 
	\label{figure:leukocyte_adhesion_cascade}
\end{figure}

\subsubsection{P-selectin}
A variety of adhesion receptors have been implicated in leukocyte endothelial interactions and are thought to play roles in recruitment, arrest, and extravasation of leukocytes. These can broadly be split into selectins and integrins, where selectins tend to be involved in the earlier stages of the leukocyte adhesion cascade and integrins in later stages (see \autoref{figure:leukocyte_adhesion_cascade}).

The selectins are a family of adhesion receptors that have been shown to mediate rolling of leukocytes. There are three known selectins; L-Selectin, E-Selectin and P-Selectin, which are encoded by three closely linked genes~\cite{Watson1990}. The integral membrane protein leukocyte receptor P-selectin is stored within the membrane of WPBs of endothelial cells~\cite{Bonfanti1989, McEver1989}, from where it is delivered to the cell surface within minutes after secretagogue-triggered exocytosis~\cite{McEver2002}. P-selectin thus plays a key early role in the inflammatory trafficking of leukocytes, being the first receptor involved in recruiting leukocytes from flowing plasma to the endothelial surface~\cite{Larsen1989}. In \autoref{leukocytes} the role of P-selectin in leukocyte extravasation and the inflammatory response is investigated via a novel approach to quantitative analysis in fluid flow assays.

At later stages in leukocyte extravasation important roles are played by the family of $\beta$2 or leukocyte integrins~\cite{Springer1990, Mayadas1993}. The genetic disease leukocyte adhesion deficiency I (LADI) is caused by a mutation of the $\beta$2 integrin gene. The increased concentration of circulating leukocytes in patients with LADI implies that $\beta$2 or leukocyte integrins play an important role in leukocyte extravasation. Patients with LADI are susceptible to frequent bacterial and fungal infections~\cite{Anderson1987}. Effective arrest of leukocytes at an inflammation site requires local activation of circulating leukocytes. The circulating leukocytes transiently interact with the activated endothelium, reducing their velocity in the direction of flow, allowing activation of their $\beta$2 integrins, leading to arrest and extravasation. This transient interaction is manifested as rolling of the leukocytes along the vessel wall in areas of inflammation~\cite{Mayadas1993, Atherton1972}.

\section{Microscopy}
\label{introduction:microscopy}
There are broadly three categories of microscopy: light microscopy, electron microscopy, and scanning probe microscopy. The oldest being light microscopy, which uses a series of lenses to create a magnified image of an object via the diffraction, reflection, or refraction of electromagnetic waves in the visible light spectrum. Developments in light microscopy since the 17th century have lead to the plethora of diverse and sophisticated instruments available today~\cite{Spector2006}. More recent alternatives to light microscopy are electron microscopy, and scanning probe microscopy. Electron microscopy images a sample by the interaction of an accelerated beam of electrons with the sample. An accelerated electron beam gives a higher resolving power than that available with visible light microscopy. This higher resolution can be achieved because excited electrons can have a shorter wavelength than visible light (see \autoref{introduction:microscopy:resolution}). The third microscopy technique is scanning probe microscopy, this diverse family of methods scans the specimen to study the surface properties via a physical probe. Scanning tunnelling microscopes (STMs) and atomic force microscopes (AFMs) are two prominent types of scanning probe microscopes.

Light microscopy is a key tool in cell biology because: the resolution is well suited to imaging cellular structures, light is relatively nonperturbing meaning that living cells can be imaged over time, and a variety of fluorescent probes can be used to  mark proteins and organelles~\cite{Thorn2016}. There are many types of light microscopy. A useful distinction is between microscopy techniques which require the use of labels such as fluorescent dyes and proteins and label-free methods. The latter label free methods  include bright field, dark field, phase contrast, differential interference contrast as well as non linear methods such as coherent anti-Stokes Raman spectroscopy (CARS) and second-harmonic generation microscopy (SHG)~\cite{Murphy2012}.

Bright-field is the most elementary microscopy technique where a sample is illuminated and imaged by its effect on the light incident on it as it absorbs, scatters, or diffracts the light. The extent of absorbency of transmitted light by the sample gives rise to varying contrast in the image. Since cells tend to be thin and transparent they often transmit light well, resulting in subtle phase differences and contrast differences. Additional optics may be added to amplify the phase shift of light induced by the cells to see extra detail in the sample. Two common techniques used to amplify this phase shift are phase contrast microscopy or differential interference contrast (DIC) microscopy.

The second class of light microscopy called fluorescence microscopy uses a different method to illuminate the sample. Fluorescence microscopy makes use of the property of chemical compounds called fluorophores. These fluorophores absorb light at a certain wavelength and emit light at another longer wavelength. Fluorophores can be introduced by the experimenter and used to fluorescently label certain proteins, cellular structures or organelles. The fluorophores may be introduced either by genetically encoding a fluorescent protein or by binding of a fluorescently labeled antibody~\cite{Thorn2016}. Fluorescence techniques are especially powerful because multiple fluorescent labels can be applied to simultaneously label specific structures. Different labelled structures can be imaged by changing the wavelength of incident light or by separating the emission spectra. Fluorophores are also detectable with a very low abundance~\cite{Ljosa2009}.

Images acquired by light microscopy methods form the basis for image processing and analysis in this thesis. Although electron microscopy was also used to image specimens at higher resolutions the resulting images were used in a more qualitative sense. In \autoref{endothelial_morphometry} endothelial cells were imaged by a confocal high-throughput microscopy approach. A structured illumination microscopy technique to obtain high resolution images of platelets was used in \autoref{platelets}. Videos of leukocytes interacting with the endothelium in \autoref{leukocytes} were acquired via a phase contrast microscopy method. A large determinant in choosing the appropriate microscopy technique was to obtain images with sufficient contrast and a resolution appropriate to the smallest structures to be imaged.

\subsection{Resolution}
\label{introduction:microscopy:resolution}
Optical resolution describes the ability of a system to resolve detail in the object that is being imaged. The optical resolution of an imaging system depends on its components such as: the lens, digital sensors, and display components. The minimum resolving power or resolution limit is defined as the minimum distance apart of two objects in order for them to be resolved as separate objects~\cite{Abbe1873}. 

For a microscope objective the limit of resolution refers to its ability to distinguish between two closely spaced Airy disks in the diffraction pattern. All optical systems have a resolution limit from diffraction occurring at the objective lens aperture. This diffraction for a circular aperture causes infinitely small points within a sample to be seen as Airy disks in the image. These Airy disks are observed as bright disks surrounded by concentric rings~\cite{Spector2006}.

Principal determinants of the resolution are the wavelength of incident particles and numerical aperture of the microscope. The numerical aperture of a lens is a measure of its ability to gather light and to resolve fine specimen detail while working at a fixed object distance. There are several equations commonly used to describe the relationship between the resolution ($r$), the wavelength ($\lambda$), the objective lens numerical aperture ($NA_{obj}$), and the condensor lens numerical aperture ($NA_{cond}$):
\begin{equation}
r=0.61\frac{\lambda}{NA_{obj}},
\label{equation:airy}
\end{equation}
\begin{equation}
r=0.5\frac{\lambda}{NA_{obj}},
\label{equation:abbe}
\end{equation}
and,
\begin{equation}
r=1.22\frac{\lambda}{NA_{obj} + NA_{cond}}.
\label{equation:resolution}
\end{equation}
These equations should not be considered as absolute but rather guidelines that can vary according to theoretical predictions of lens behaviour, the difference in multiplication factor between \autoref{equation:airy} and \autoref{equation:abbe} demonstrates this.

Imaging at an appropriate magnification and resolution are large determining factors when choosing the optical setup for an experiment. Spatial resolution requirements in an imaging system are defined by the separation of objects in the sample.

\subsection{Confocal microscopy}
\label{introduction:microscopy:confocal}
In 1957 Marvin Minsky patented the concept for a stage-scanning confocal optical system~\cite{Minsky1988}. A confocal microscope offers several key advantages over conventional wide-field microscopy, including: the ability to control the depth of field, eliminating out of focus image-degrading light, and being able to collect sequential optical sections enabling the reconstruction of three-dimensional structures.

Viewing a sample with a conventional wide-field fluorescence microscope floods the sample evenly in light. All fluorophores in the specimen are excited and the resulting fluorescence is detected at the microscope camera. This gives a plane of in-focus light, as well as degrading contributions from out of focus planes over the depth of field. A confocal microscope on the other hand only images light in focus, this is achieved using point illumination of the sample and a pinhole aperture in a conjugate plane in front of the detector. The pinhole is positioned in a plane conjugate to the focal point of the lens thus it is a confocal pinhole.

As can be seen in \autoref{figure:confocal_schematic} the grey ray paths that are not from the in-focus focal plane do not pass through the detector pinhole aperture so are not imaged. Only light from fluorescence axially close to the focal plane is detected. Confocal microscopes have an image resolution in the axial depth plane much finer than a wide-field microscope. Since a pinhole aperture blocks much of the light from reaching the detector longer exposures are often required to obtain images with a suitable signal-to-noise ratio. Longer exposures are also required for suitable signal-to-noise because a confocal laser scanning microscope is point scanning. The pixel dwell time is much less than equivalent in a (parallel) widefield system in which all camera pixels receive photons for the duration of the total image exposure. Additionally, to increase the signal intensity the confocal optical setup often includes a photomultiplier tube before the light detector to amplify the light signal (see \autoref{figure:confocal_schematic}).

\begin{figure}[htbp!]
	\centering
	\includegraphics[width=1.0\textwidth]{drawing:confocal_schematic}
	\caption[Schematic of confocal microscopy optics]{Schematic to illustrate the optics of an epi-fluorescence confocal microscope, with a fluorophore that absorbs blue wavelength light and emits in the green wavelength. Pinhole apertures at the light source and detector allow the focus of light rays only at the point in focus on the specimen to pass to the detector. The dashed grey paths show light that is not in the focal plane of the image so does not not travel through the pinhole aperture to the detector and is thereby eliminated in the final image.}
	\label{figure:confocal_schematic}
\end{figure}

An important feature of confocal microscopes is the dichroic or dichromatic mirror. In order to excite the fluorophores in the sample in \autoref{figure:confocal_schematic} the light incident on the sample needs to be blue, a longer wavelength green light is then emitted by the fluorophores. A dichroic mirror reflects light shorter than a certain wavelength, and only transmits light longer than that wavelength. In this way at the detector only the emitted green light is observed and not the scattered blue light.

The schematic in \autoref{figure:confocal_schematic} is capable of imaging a single point source, but in order to obtain a full image from point illumination the sample is scanned in a regular raster. Two methods were originally proposed for obtaining this raster image, either by moving the sample or using a system of oscillating scanning mirrors~\cite{Minsky1988}. The oscillating mirrors are the more common method since they allow for more precise control in the horizontal plane and do not disturb the sample. A series of two-dimensional rasters can also be acquired at sequential focal depths and reconstructed into a three-dimensional \emph{z-stack}, or displayed as a maximum intensity projection. The obtainable resolution in the vertical direction will usually be less than that in the horizontal plane and will depend on he wavelength of the light and the numerical aperture of the lens.

The development of lasers and the first confocal laser scanning microscope,  along with steady improvements in raster scanning has made the confocal microscopy a hugely practical and widespread technique in the life sciences~\cite{Sheppard1977}. In this body of work confocal microscopy has been used extensively with an automated high-content screening approach as described \autoref{endothelial_morphometry}.

\subsection{Structured illumination microscopy}
\label{introduction:microscopy:structured_illumination_microscopy}
The family of technically innovative methods of increasing the limits of optical resolution by at least a factor of two are collectively known as super-resolution microscopy techniques. Super-resolution microscopy is a form of light microscopy that allows samples to be imaged at resolutions higher than the resolution limit as stated by Ernst Abbe in 1873 (see \autoref{equation:abbe})~\cite{Abbe1873}. These techniques have increased the maximum attainable resolution from light microscopy from $\sim$\SI{250}{\nano\meter} to $\sim$\SI{10}{\nano\meter}~\cite{Galbraith2011}. Advances in super-resolution technologies are either based on tailored illumination, nonlinear fluorophore responses, or the precise localisation of single molecules~\cite{Schermelleh2010}. A commonly used method in super-resolution microscopy is structured illumination microscopy (SIM).

The super-resolution technique SIM images a sample using a series of sinusoidal or striped illumination patterns~\cite{Gustafsson2000}. Typically the striped illumination is produced by a laser passing through a grating and projected by the objective lens onto the sample. When illuminated with this patterned light an interference pattern called a Moir\'e fringe is produced, which shifts high spatial frequency information to lower frequencies that are detectable by the microscope. \autoref{figure:introduction:sim:moire} shows a Moir\'e fringe arising from two superimposed patterns, in the area of overlap between the two patterns a lower frequency coarse pattern can be observed. By illuminating a sample with patterned light in this way high frequency information beyond the resolution limit of the objective lens is shifted to lower frequencies.

\begin{figure}[htbp!]\centering
\hspace*{\fill}
	\begin{subfigure}[b]{0.33\linewidth}
		\centering
		\includegraphics[width=\linewidth]{drawing:moire_fringe}
		\caption{}
		\label{figure:introduction:sim:moire}
	\end{subfigure}
	\hfill
	\begin{subfigure}[b]{0.33\linewidth}
		\centering
		\includegraphics[width=\linewidth]{platelet}
		\caption{}
		\label{figure:introduction:sim:platelet}
	\end{subfigure}
\hspace*{\fill}
\caption[Structured illumination microscopy]{In image (a) the Moir\'e effect can be seen from two sets of striped lines rotated and offset, this is a key concept behind structured illumination microscopy. The image in (b) shows a platelet imaged both under conventional wide-field light microscopy method in the upper left region and a structured illumination microscopy technique in the lower right region, adapted from Westmoreland et al., 2016~\cite{Westmoreland2016}. Scale bar: \SI{1}{\micro\meter}}
\label{figure:sim}
\end{figure}

To construct an image using the high frequency information contained in Moir\'e fringes a series of images with different Moir\'e fringes are acquired. 
This is achieved by capturing images of the sample under illumination with different known patterns, the high frequency information encoded in the Moir\'e patterns can be seperated and shifted back to its true origin in Fourier space. Since the illumination pattern is known and can be mathematically described, it can be deconvolved from the Moir\'e pattern to observe higher frequency information that is otherwise irresolvable. A high-resolution of the underlying structure can be acquired by using algorithms to deconvolve and combine the acquired images.

In this work SIM has been used to image platelets and their dense-granules to form the foundation of a new tool for diagnosis of patients with dense-granule disorders. The maximum obtainable resolution by linear SIM is limited to a two-fold increase on that available by conventional microscopy since the periodicity of the illumination pattern is created by diffraction~\cite{Gustafsson2000}. SIM increases the image resolution to $\sim$\SI{100}{\nano\meter}~\cite{Huang2010}. This increased resolution was necessary to image platelet dense-granules for that cannot easily be resolved through conventional light microscopy (see \autoref{figure:introduction:sim:platelet}).

\subsection{Phase contrast microscopy}
\label{introduction:microscopy:phase_contrast_microscopy}
In cell biology many specimens are highly transparent with little natural pigmentation. The differences in light absorption between different cellular structures are minimal, and much of the contrast and detail between structures is difficult to detect by conventional microscopy. Although the various cellular organelles tend not to be good light absorbers they can have considerable variations in refractive index~\cite{Murphy2012}. Light travelling through these structures with different refractive indices is slowed by differing amounts, this can be measured as a phase shift in the light waveform. Phase contrast microscopes allow us to visualise differences in optical path length, which is the product of the geometric path length and the refractive index of the medium. 

The interaction of light travelling through a medium causes changes in the amplitude and phase of the wave dependent on properties of the medium. The amplitude changes occur from scattering and absorption of light, which is visible as the brightness of the light. Phase changes in the light are not visible without specialised equipment. Phase contrast microscopes feature an optical system that transforms differences in the phase of non-diffracted light, and light diffracted by a specimen into contrast differences in the image (see \autoref{figure:phase_contrast_schematic}). In this way transparent structures that may be otherwise unobservable are visible as if they had been optically stained. A phase contrast microscope measures the phase shift or retardation caused by the specimen in the optical path relative to background light. The velocity ($v$) of light in a medium is given by  $v=c/n$, where ($c$) is the speed of light in a vacuum, and ($n$) the refractive index of the medium. Light waves traveling through an object travel slower and emerge phase shifted relative to the background rays.

\begin{figure}[htbp!]
	\centering
	\includegraphics[height=0.7\textheight]{drawing:phase_contrast_schematic}
	\caption[Schematic of phase contrast microscopy optics]{Schematic to illustrate the optics of a phase contrast microscope. The light paths for diffracted and non-diffracted light through the condenser annulus and phase plate are shown, a plan view of the phase annulus and phase ring is also shown.}
	\label{figure:phase_contrast_schematic}
\end{figure}

The specialised phase contrast microscope has a phase annulus that generates a hollow cone of non-diffracted illumination. This hollow cone of light is projected onto the back focal plane of the objective (see \autoref{figure:phase_contrast_schematic}). The phase plate in the objective of the phase contrast microscope performs two functions: it phase shifts the light by a quarter wavelength ($\lambda/4$), and it attenuates or absorbs the non-diffracted light by \SIrange{70}{80}{\percent}~\cite{Spector2006}. The light is focused on the image plane and in regions where the specimen is present, constructive interference occurs between the non-diffracted and diffracted light, this results in an increase in contrast compared to regions not containing the specimen.

Phase contrast microscopes are well suited to imaging thin specimens (\textless \SI{5}{\micro\meter}), for thicker specimens the \emph{halo-effect} can reduce the image quality~\cite{Spector2006}. The formation of halos around objects can arise due to refractive gradients passing the non-diffracted light outside the phase contrast ring and makes it difficult to determine the edges of cell structures~\cite{Otaki2000}.

In this work phase contrast microscopy has been used to acquire multiple sequential frames in a video sequence of leukocytes flowing over an endothelial monolayer. Interactions of the leukocytes with the monolayer were quantified and used to infer biological function, for this live cell assay phase contrast microscopy was well suited, this is described further in \autoref{leukocytes}.

\section{Digital image processing}
\label{introduction:image_processing}
The earliest discovered fossilised eyes date from the lower Cambrian period approximately 540 million years ago~\cite{Parker2011}. From a patch of photoreceptors the visual system has evolved into a complex and essential sensory system. In humans the visual system is adept at intuitively perceiving and interpreting visual information to understand a scene. It has the ability to seamlessly filter irrelevant information such as noise or illumination variations within a scene. Computer vision uses techniques from fields such as image processing, pattern recognition and machine learning to replicate functions of the human visual system. In computer vision images are stored in computer memory as digital raster images, containing a fixed number of rows and columns of pixels (picture elements).

Computer vision and image processing are becoming increasingly relevant in biological tasks such as in measurement, evaluation, and interpretation of microscopy images. These computerised approaches further biological insight by performing image processing, segmentation, recognition, and interpretation. As modern biology shifts towards experiments involving large, highly quantitative data sets manual approaches to image interpretation are becoming antiquated. With such large experimental data sets an expert visual inspection and annotation is not feasible. Furthermore, human inspection is not suited to detecting subtle population differences over tens, or hundreds of thousands of images.

Nascent computerised image processing and analysis have become a vital part of the experimental workflow. An automated or semi-automated computational approach to image processing is able to process large data sets in a fraction of the time it would take to perform the analysis manually. Further additional benefits to computerised image analysis over a manual visual inspection include:
\begin{itemize}
\item providing unbiased highly quantitative scoring,
\item being highly reproducible,
\item being able to handle large data sets,
\item being able to detect subtle population differences,
\item eliminating tedious manual labour, and
\item being able to extract and simultaneously collect multiple features~\cite{Jones2006}.
\end{itemize}

The detection and measurement of cellular structures in microscopy images is essential to cell biology. Segmentation and feature extraction are also important topics in computer vision and image processing. Thus developing algorithms to automatically identify and measure structures of interest within biological images provides the basis for many productive collaborations between biologists and computational image processing algorithm developers. As yet general purpose image analysis methods have not been established due to the diversity of microscopy and image acquisition methods employed in cell biology. Image analysis approaches remain tailored to the specific experimental setup, however as artificial intelligence (AI) and machine learning methods continue to develop more generic methods could become viable.

A general approach to experimentation in life sciences is to investigate defined research questions or hypotheses via quantitative analyses of experimental data. A single experiment or a series of experiments may be constructed to investigate a hypothesis. For a hypothesis to be accepted results must be reproducible over multiple experiments and corroborated by alternative experimental methods.

\begin{figure}[htbp!]
	\centering
	\includegraphics[width=1.0\textwidth]{drawing:experimental_workflow}
	\caption[Experimental workflow in life science imaging]{A conceptual loop depicting an experimental workflow in life sciences involving microscopy imaging. The inverse correlation between data volume and information complexity is shown. Adapted from Prodanov and Verstreken, 2012~\cite{Prodanov2012}.}
	\label{figure:endothelial_morphometry:introduction:experimental_workflow}
\end{figure}

The experimental loop displayed in \autoref{figure:endothelial_morphometry:introduction:experimental_workflow} shows typical stages in experiment design and analysis. The top row of processes deals with the biological hypothesis and experimental setup, and the bottom row of processes are relating to image analysis and interpretation of the results. The bottom row of processes comprises the image analysis workflow, which converts the raw data into a form that can quantitatively address the initial hypothesis. Collection of data, aggregation of data and additional analysis allows for interpretation of the data. The loop is self-perpetuating since often the investigation of a hypothesis leads to investigation of further hypotheses.

This section gives some background to digital images, pre-processing, segmentation, morphological processing and feature extraction methods that have been developed in this work. These methods and their applications are described in more detail in \autoref{endothelial_morphometry}, \autoref{platelets}, and \autoref{leukocytes}.

\subsection{Image acquisition and digitisation}
\label{introduction:image_processing:image_acquisition}
Digital images are acquired by the conversion of information transmitted as electromagnetic radiation into electrical signals that are measured and stored as values in a digital format. Incident electromagnetic radiation in the form of photons are converted at photosensitive detectors on a charged-coupled device (CCD) into electrons generating a voltage. The voltage can be measured and stored as a number indicating the intensity of incident light.

The intensity values measured by the CCD are stored as a matrix of values in a digital image, this can be processed by a computer and displayed. The formation of a digital image requires careful consideration of the image acquisition approach and method of digitisation. Optimisation of image acquisition and digitisation ensures the greatest efficacy of subsequent image processing and data analysis. Image analysis can rarely extract good data from poor quality images, and it is often easier to improve the data quality by optimising the image acquisition than by performing additional image processing~\cite{Roeder2012}.

An important factor in image digitisation is to consider the sampling. Sampling is the process of converting a continuous analogue signal into a discretised finite set of values. Spatial sampling is given by the image resolution, whilst amplitude or intensity sampling is given by the image bit-depth and is known as quantisation. The effects of spatial and intensity sampling are illustrated in \autoref{figure:introduction:sampling}.

\begin{figure}[htbp!]\centering
	\begin{subfigure}[b]{0.32\linewidth}
		\centering
		\includegraphics[width=\linewidth]{nucleus_res128}
		\caption{}
		\label{figure:introduction:sampling:res128}
	\end{subfigure}
	\begin{subfigure}[b]{0.32\linewidth}
		\centering
		\includegraphics[width=\linewidth]{nucleus_res64}
		\caption{}
		\label{figure:introduction:sampling:res64}
	\end{subfigure}
	\begin{subfigure}[b]{0.32\linewidth}
		\centering
		\includegraphics[width=\linewidth]{nucleus_res32}
		\caption{}
		\label{figure:introduction:sampling:res32}
	\end{subfigure}
	\begin{subfigure}[b]{0.32\linewidth}
		\centering
		\includegraphics[width=\linewidth]{nucleus_bit16}
		\caption{}
		\label{figure:introduction:sampling:bit16}
	\end{subfigure}
	\begin{subfigure}[b]{0.32\linewidth}
		\centering
		\includegraphics[width=\linewidth]{nucleus_bit4}
		\caption{}
		\label{figure:introduction:sampling:bit4}
	\end{subfigure}
	\begin{subfigure}[b]{0.32\linewidth}
		\centering
		\includegraphics[width=\linewidth]{nucleus_bit2}
		\caption{}
		\label{figure:introduction:sampling:bit2}
	\end{subfigure}
\caption[Sampling in digitisation]{A depiction of the effects of sampling spatially and by intensity amplitude in digitisation of images. The top row of images show the effect of down-sampling the spatial resolution, where image resolutions are: (a) $128 \times 128$, (b) $64 \times 64$, and (c) $32 \times 32$. Images have been resized to illustrate the loss of detail. The bottom row of images show the impact of quantisation where the number of intensity levels are: (a) 16, (b) 4, and (c) 2.}
\label{figure:introduction:sampling}
\end{figure}

Image resolution depends on the number of pixels recorded, the optics of the system, and the electronic properties of the CCD. The image resolution may be limited by the microscope optics (see \autoref{introduction:microscopy:resolution}) or by the spatial sampling rate. The higher the image resolution the truer the digital representation is to the original continuous signal. A low resolution image can distort the real observations. In the context of microscopy, resolution is the conversion of physical space into a set of equally spaced grid points on an image~\cite{Russ2006}. The finer the two dimensional grid the greater the detail that can be observed in the image.

Image quantisation determines the set of allowed numbers to represent intensity values of pixels in an image~\cite{Pawley2006}. The quantisation level is expressed as bit-depth, where the greater the bit-depth the more allowed light intensity values there are. Having more allowed intensity values allows for the representation of more shades and colours in the image. Frequently used raster image bit-depths are 8-bit and 16-bit; an 8-bit image allows for $2^{8}$ or 256 intensity values and a 16-bit image allows for $2^{16}$ or 65536 intensity values. Some microscopes also use 12-bits per channel, giving $2^{8}$ or 4096 possible grey values. The number of quantisation levels should be high enough for human perception of fine shading details in the image.

The process of digitisation leads to noise in images manifest as variance in the intensity values above and below the true intensity values of the signal~\cite{Waters2009}. Noise adds a level of uncertainty to the accuracy of measurements due to imprecision in measurements of pixel intensity values. Poisson noise or shot noise arises in an image from the fact that intensity values are obtained from measurements of photons arriving at a CCD. These events have an intrinsic uncertainty as they are stochastic quantum events limited by Poisson statistics~\cite{Pawley2000}. Poisson noise cannot be reduced or eliminated, although its effect can be minimised by increasing the number of signal photons counted, the Poisson noise then becomes a smaller fraction of the signal. As the signal increases relative to the noise level, measurements of the signal become increasingly more precise. The precision of quantitative microscopy measurements is therefore limited by the signal-to-noise ratio (SNR) of the digital image.

\subsection{Image pre-processing}
\label{introduction:image_processing:image_pre-processing}
Pre-processing of images occurs at the lowest level of abstraction where both the input and output are intensity images~\cite{Sonka2007}. The necessity of this stage in the image analysis workflow is dependent on the quality of images from acquisition. Where possible it is preferable to improve image acquisition rather than create elaborate pre-processing pipelines. Nonetheless, pre-processing of images can often be useful to suppress information that is not relevant to the particular image analysis.

The aim of pre-processing is to improve the image data by enhancing certain image features, or to correct for defects in acquired images. These defects may arise for example due to imperfect detectors, limitations on microscopy optics, or non-uniform illumination. A useful categorisation of image pre-processing splits methods into four basic types: pixel brightness transformations, geometric transformations, local neighbourhood pre-processing and image restoration~\cite{Sonka2007}. This categorisation is based on the size of the pixel neighbourhood from a single pixel in pixel brightness transformations to the whole image in image restoration. In this work pixel brightness transformations and local pre-processing were mainly used so are described further.

\paragraph{Pixel brightness transformations} A brightness transform modifies pixel values depending on the properties of the pixel itself. A useful grey-scale transform for contrast enhancement can be achieved using the \emph{histogram equalisation} technique. The aim of this is to distribute brightness levels over the intensity range of the image. Also in this group of brightness transformations are the adaptive histogram enhancement, and contrast limited adaptive histogram enhancement~\cite{Ketcham1974}. Pixel brightness transforms involving histogram enhancements are especially useful for increasing contrast in displayed images to better observe image features.

\paragraph{Filtering} Local pre-processing operations by filtering use a neighbourhood of surrounding pixels in a kernel to transform the intensity of pixels within an image. This kernel or convolution mask usually comprises a rectangular neighbourhood with an odd number of pixels in rows and columns, thus enabling specification of a central pixel in the kernel. The size and the shape of kernel will depend on the objects in the image to be processed. If for example the objects are large relative to noise as is the case in \autoref{figure:introduction:sampling} then a small kernel is suitable and  will not significantly degrade the large structures.

Filtering methods using a kernel  may be used for smoothing, where the principle use of image smoothing kernels is to suppress image noise. This is often achieved by calculating an average brightness value of a neighbourhood. The median filter for example considers the neighboring pixels, sorts them to find the one with the median value and assigns that value to the pixel of interest. A similar operation called the \emph{mean filter} averages the values of neighbourhood pixels and assigns that to the pixel of interest. The Gaussian filter is also useful for smoothing images using a Gaussian kernel. Smoothing methods that can reduce noise while preserving texture patterns and edges in an image are often more useful in improving the outcomes of object detection and segmentation, however are generally computationally more expensive~\cite{Sonka2007}. Examples of edge-preserving smoothing methods that smooth textures whilst retaining sharp edges are: the Bilateral filter, the Guided filter, and Anisotropic diffusion using the heat equation~\cite{Pal2015}.

\subsection{Segmentation}
\label{introduction:image_processing:segmentation}
Image segmentation is a crucial prerequisite step prior to feature extraction and analysis in most image analysis workflows. The goal of segmentation is to partition an image into multiple homogeneous segments, or sets of pixels. This partitioning changes the representation of an image into a more meaningful form. Segmentation is often used to identify certain distinct structures within an image, for example in cell biology these could be nuclei or cells. This is achieved by way of assigning a label to every pixel within an image such that pixels with the same label share certain characteristics. In the most simplistic case these labels can distinguish between foreground and background pixels~\cite{Russ2006}.

There are many approaches to segmentation which are being constantly developed and improved upon including: clustering methods, compression-based methods, edge detection and region growing methods. Two of the most fundamental and commonly used methods of segmentation are thresholding and the watershed transform.

\paragraph{Thresholding} The simplest segmentation process is grey-level thresholding, where an intensity value or threshold is determined to segment objects and background. Thresholding is computationally inexpensive and fast to compute~\cite{Sonka2007}. A threshold can be calculated and applied globally or locally to an image, where local thresholding can account for differences in the intensities of pixels in different regions of the image. A number of algorithms have been developed for automatic selection of a threshold value. Histogram shape analysis is often used to calculate a threshold value. For example the mode method for thresholding bi-modal histograms finds the highest local maxima and then detects the minimum between two maxima. The choice of thresholding method will depend on the specific image data and it is usually worth experimenting with many methods of obtaining a threshold value such as Otsu thresholding~\cite{Otsu1979,Kittler1986} or moment preserving thresholding~\cite{Tsai1985} to determine the most appropriate method.

\paragraph{Watershed} A watershed transform is a method for segmentation based conceptually on a hydrological watershed. In a hydrological sense a watershed is the line dividing adjacent catchment basins. An image can also be visualised as a topographic surface, where the \emph{z-axis} represents altitudes. Peaks and ridges in the image correspond to high intensity and watershed regions and low intensity regions are catchment basins. Watershed segmentation works to identify and label catchment basins in an image. There are multiple algorithms that have been developed to achieve this~\cite{Roerdink2000}. A one dimensional example of watershed segmentation is shown in \autoref{figure:introduction:watershed}

\begin{figure}[htbp!]\centering
	\begin{subfigure}[b]{0.48\linewidth}
		\centering
		\includegraphics[width=0.9\linewidth]{drawing:watershed_1}
		\caption{}
		\label{figure:introduction:watershed_1}
	\end{subfigure}
	\begin{subfigure}[b]{0.48\linewidth}
		\centering
		\includegraphics[width=0.9\linewidth]{drawing:watershed_2}
		\caption{}
		\label{figure:introduction:watershed_2}
	\end{subfigure}
\caption[Watershed transform 1D]{An example of a watershed segmentation in one-dimension. The plot in (a) is a greylevel intensity profile of image data. In image (b) watershed segmentation has been performed showing the local minima catchment basins and the watershed lines.}
\label{figure:introduction:watershed}
\end{figure}

There are two basic approaches to performing a watershed transform. The original computationally intensive method finds a downstream path to a local minimum of image topology for each pixel in the image. The catchment basin is then the set of pixels for which the downstream paths all end up in the same altitude minimum. A second approach considers the flooding of the relief and imagines the progressive filling of the valleys, but to a point that the flooding does not spill into the next valley. The flooded regions are labelled as a specific segmented objects in the relief. Vincent et al., 1991  presented an efficient algorithm to perform this based on sorting pixels in increasing order of value, followed by a flooding with a breadth-first scanning of all pixels in their sorted order~\cite{Vincent1991}.

\subsection{Morphological processing}
\label{introduction:image_processing:morphological_processing}
In image processing morphological processing is a theory and group of techniques for performing transformations on geometrical structures, based on set theory, lattice theory, topology, and random functions~\cite{Goutsias2000}. These techniques involve changing the shape of objects in binary or greyscale images using morphological rules from mathematical set theory. Morphological operations on an image are relations of two sets; the image and a structuring element.

Morphological techniques manipulate an image with a kernel or template called a structuring element. The structuring element is positioned at each pixel location within an image and operations performed on the neighbourhood. An output image is generated from the interaction of the structuring element with the image. There are many morphological operations some examples include: dilation, erosion, opening, closing, skeletonisation and distance transforms.

\paragraph{Binary dilation and erosion}
The fundamental binary morphological operations are dilation and erosion. These operate on binary images comprised of black and white pixels. In binary images one class of pixels is considered foreground and the other background. From the basic morphological operations of dilation and erosion further more complicated operations arise such as opening, closing, and shape decomposition.

\begin{figure}[htbp!]\centering
	\begin{subfigure}[b]{0.32\linewidth}
		\centering
		\includegraphics[width=\linewidth]{nucleus_binary}
		\caption{}
		\label{figure:introduction:nucleus_binary}
	\end{subfigure}
	\begin{subfigure}[b]{0.32\linewidth}
		\centering
		\includegraphics[width=\linewidth]{nucleus_binary_dilated}
		\caption{}
		\label{figure:introduction:nucleus_dilated}
	\end{subfigure}
	\begin{subfigure}[b]{0.32\linewidth}
		\centering
		\includegraphics[width=\linewidth]{nucleus_binary_eroded}
		\caption{}
		\label{figure:introduction:nucleus_erosion}
	\end{subfigure}
\caption[Binary dilation and erosion]{An example of binary dilation and erosion performed on a thresholded image of nuclei. Image (a) is the thresholded nuclei, image (b) is the result after performing binary dilation on (a), and image (c) is the image after performing binary erosion on (a).}
\label{figure:introduction:nucleus_dilation_and_erosion}
\end{figure}

A dilation is an isotropic expansion that can be used to fill small holes or join objects together resulting in an increased object size. Conversely an erosion acts is an isotropic shrinking that reduces the size of objects in the image or if sufficiently small removes them. These operations are demonstrated in \autoref{figure:introduction:nucleus_dilation_and_erosion}. Erosion and dilation are not inverse transforms meaning that if an image is eroded and then dilated the resulting image differs from the original image, instead a simplified version of the original image is obtained with less detail. The process of erosion followed by dilation is called opening and is useful to eliminate specific image details smaller than the structuring element whilst maintaining the general shape of the objects in the image. The process of dilation followed by erosion is known as closing. This is useful to connect objects that are close to each other and fill small holes in objects.

\subsection{Region description and features}
\label{introduction:image_processing:morphological_processing}
Segmentation and morphological processing are important steps to constructing homogeneous image regions. The extraction of semantic information from these regions is important in understanding the image data. This requires a description of that region, which can be achieved by generating a numeric feature vector. This numeric vector may contain features relevant to shape, location or intensity that describe the region. There are many shape descriptors however there is no generally accepted method of shape description. In this work a range of shape descriptors have been used, these are described further in \autoref{endothelial_morphometry:morphometric_measurements} and \autoref{platelets:morphometric_measurements}.

The description of a region is first dependent on its identification, a useful method for region identification is to label each region with a unique integer pixel value. This process is known as labelling or colouring, in which the largest integer number also gives the number of regions in the image. For each labelled region a numeric vector of multiple features can be measured.

Region based shape descriptors may be from simple geometric descriptions such as area, eccentricity, and direction. Other shape descriptors such as statistical moments interpret a greyscale image function as a probability density of a 2D random variable, the properties of which can be described by moments. Shapes can also be described by regional decomposition into simpler shapes, for example by thinning to obtain a skeleton. Pixel intensity features can include the mean pixel value of a region, the maximum, minimum, or standard deviation for example.

\subsection{Segmentation performance evaluation}
\label{introduction:image_processing:performance_evaluation}
The evaluation of image segmentation is an important and often overlooked stage of image analysis. Despite there being many algorithms and significant literature about image segmentation techniques, there is no universal agreement on how to evaluate the performance of image segmentation algorithms~\cite{Benes2015}. For any given segmentation problem there are multiple possible segmentation algorithms that could be used, finding the optimal method and parameter choice requires a choice of evaluation metrics suitable for the data use.

A helpful starting point to assess the performance of a segmentation algorithm is to review by eye the segmentation contours overlaid on the raw image. This human evaluation is tedious and limits the number of comparisons to a small sample size~\cite{Zhang2008}. Often more thorough quantitative assessment is required to detect subtle differences between segmentation methods. In image processing and computer vision a comparison to a gold standard or reference segmentation data set is a commonly used and instructive method of performance evaluation.

The term gold standard refers to a benchmark that is available under reasonable conditions. The ground truth differs from the gold standard in that it is obtained by a set of measures that is known to be more accurate than the measurements of the system being tested. A ground truth could for example be obtained by performing segmentation of higher resolution and higher magnification microscopy images. A gold standard on the other hand is not an objectively perfect set, but merely the best available~\cite{Cardoso2014}. A gold standard is acquired from the most accurate segmentation procedure, which in this case is generated by manually hand-labelling or semi-automatic labelling images.

Image segmentation is often an ill defined problem, meaning there may not be a single gold standard segmentation. It is therefore useful to compare against multiple perceptually consistent interpretations of an image. These might be for example hand segmented by several experts in the field. Hand segmentation of a gold standard is however a labour intensive process and this is often not feasible~\cite{Unnikrishnan2005}. The evaluation of a segmentation algorithms performance and generation of a gold standard data is also complicated by inconsistent data, since images of microscope samples vary in quality and character.

A useful way of evaluating the performance of a automatically segmented image to a gold standard is to use overlap ratio measures. These give a similarity measure between the pixel sets of segmented objects. In \autoref{endothelial_morphometry} the Dice coefficient and Jaccard index have been used extensively for evaluation. In other instances segmentation performance may be evaluated in a binary sense, for example when counting objects it can be useful to use binary classification tests to find the sensitivity and specificity. The sensitivity also called the true positive rate, the recall, or probability of detection in some fields measures the proportion of positives that are correctly identified as such. The specificity also called the true negative rate measures the proportion of negatives that are correctly identified as such~\citeParikh2008}.

\section{Thesis overview}
\label{introduction:overview}
An initial aim of this work was to further develop an image processing tool that had previously been used to quantify the morphometric features of a large population of the endothelial organelles Weibel-Palade bodies (WPBs)~\cite{Ferraro2014}. The computational tool discussed in \autoref{endothelial_morphometry} is an extension to this method applied to large scale morphometric studies of WPBs and endothelial cells more generally. Segmentation of endothelial cells and assignment of organelles to these cells enables both whole population level and cellular level analysis. The study of endothelial cells and WPBs under different physiological and pathophysiological conditions can be used to gain an understanding of underlying haemostatic processes. Additional data mining, statistical techniques, and machine learning of cellular and organelle features provides further biological insight.

The workflow described in \autoref{endothelial_morphometry} has also been applied extensively in high-throughput studies of the morphometry of WPB exocytic sites. These WPB `exocytic sites' reflect the presence on the endothelial plasma membrane of a fusion pore as seen by fluorescent labelling of endothelial cells with vWF internally and externally. Through counting absolute numbers of exocytic sites, the number of exocytic sites per cell and the areas of exocytic sites, these analyses can be used to garner information about von Willebrand Factor (vWF) release that is complementary to and potentially more sensitive than the commonly used enzyme-linked immunosorbent assay (ELISA) method. These studies are proving to be useful in understanding the cellular mechanisms that are employed by endothelial cells in controlling vWF release.

The release of vWF into the blood stream occurs when WPBs fuse with the plasma membrane, whereupon the vWF unfurls to form long platelet-capturing strings. These platelet-decorated strings perform a vital role in primary haemostasis. Within platelets are granules containing bioactive pro-haemostatic molecules. Genetic mutations can reduce the number or capacity of platelet granules affecting their functional ability. Genetic abnormalities such as in Hermansky-Pudlak syndrome (HPS) lead to abnormalities in numbers or capacity of platelet granules resulting in bleeding disorders. Diagnostic imaging of platelets to count their dense-granules has traditionally required electron microscopy, due to the small size of the platelet granules. Dense-granules are approximately \SI{150}{\nano\metre} wide and light microscopy has historically been limited to resolutions of \SIrange{200}{300}{\nano\metre} depending on the technique. However, developments in super-resolution microscopy now allow sufficient resolution in light microscopy to effectively image platelet granules. An automated image analysis pipeline discussed in \autoref{platelets} provides a quantitative, unbiased, rapidly acquired data set, that forms the basis for a platelet dense-granule disorder diagnostic tool. Using the antigen CD63 as a marker, granules were segmented and assigned to their respective platelets. A proof-of-principle data set with seven healthy controls and three patients with platelet granule abnormalities caused by HPS has shown the effectiveness of this technique.

Finally, in \autoref{leukocytes} a workflow developed for video analysis of leukocyte extravasation is discussed. Leukocyte recruitment is part of the innate immune response whereby stages in the leukocyte adhesion cascade: initial capture, rolling, slow rolling and firm adhesion, can be imaged under flow in vitro as a time series. Previously, analysis from such videos has involved manually counting leukocytes that interact by slow rolling or firmly adhering. The method developed can accurately detect leukocytes in each frame and track their movement between frames. Information about the trajectories of each leukocyte over an endothelial cell surface can then be acquired and data mining and analysis performed. This extra quantitative data from each experiment allows for a more thorough analysis of the leukocyte adhesion cascade than a manual counting approach.

The development of this suite of image processing tools has helped to build a multifaceted picture of the endothelial processes of haemostasis, thrombosis and inflammation. The image processing tools together with the analysis of their outputs allows for the extraction of a great deal of quantitative information to underpin significant advances in the mechanisms that are at the heart of haemostasis, and of the initiation of inflammation.
