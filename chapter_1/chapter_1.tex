\chapter{Introduction}
\ifpdf
	\graphicspath{chapter_1/figs/}
\fi

\nomenclature[z-HUVEC]{$HUVEC$}{Human Umbilical Vein Endothelial Cell}
\nomenclature[z-WPB]{$WPB$}{Weibel Palade Body}                                                   % first letter A is for Roman symbols

%\mynote{Hey! I have a note}
The overarching objective of this work is to advance our understanding of the cell biology of haemostasis, thrombosis and inflammation. The aim of this specific project is to develop a suite of image and video analysis approaches to quantify and aid in evaluation of data obtained from investigations into endothelial, haemostatic and inflammatory processes. Increasingly sophisticated and diverse microscopy is being used in cardiovascular experimentation. Automated, quantitative image analysis approaches are becoming increasingly necessary to harvest data from the often large image based data sets. This project aims to develop tools for extracting quantitative output from images, which will be useful not only to this research team, but more generally in the field. 

An initial aim of this project was to further develop an image processing tool that has been used to quantify the morphometric features of a large population of Weibel-Palade bodies (WPBs). This computational tool, discussed in section~\ref{singleCellAnalysis}, can be applied to large scale morphometric studies of WPBs under different physiological and pathophysiological conditions to gain an understanding of underlying haemostatic processes. Data mining of WPB morphometric features at a cellular level and whole population level could then be employed along with statistical, machine learning and visualisation techniques to gain further biological insight.

An adaptation of the WPB morphometric computational analysis tool (discussed in section~\ref{exitSites}), has been applied to a high-throughput study of the morphometry of WPB exocytic exit sites. A WPB exocytic exit site is observed on the endothelial plasma membrane as a fusion pore, by immunofluorescent labelling of endothelial cells with vWF internally and externally. Through counting absolute numbers of exit sites, the number of exit sites per cell and the areas of exit sites, this analysis can be used to garner information about von Willebrand Factor (vWF) release that is complementary to and potentially more sensitive than the commonly used enzyme-linked immunosorbent assay (ELISA) method. This study is also proving to be useful in understanding the cellular mechanisms that are employed by endothelial cells in controlling vWF release.

The release of vWF into the blood stream occurs when WPBs fuse with the plasma membrane, the vWF unfurls to form long platelet-capturing strings. These platelet-decorated strings perform a vital role in primary haemostasis. An unbiased image analysis workflow has been developed to segment and quantify flourescent stained images of nuclei, vWF strings and platelets under flow from human umbilical vein endothelial cells (HUVECs). A description of this automated image processing workflow is given in section~\ref{vWFstrings}. Currently, published analyses of strings have focused on their number, and how this changes in different physiological conditions. Our new workflow will capture both number and length and we have begun to uncover how this also changes in response to endothelial physiology. 

The vWF strings play an important role in capture of platelets which contribute to haemostasis. Within platelets are granules containing bioactive pro-haemostatic molecules.  Genetic mutations can reduce the number or capacity of platelet granules affecting their functional ability. Genetic abnormalities such as in Hermansky-Pudlak syndrome (HPS) lead to abnormalities in numbers or capacity of platelet granules resulting in bleeding disorders. Diagnostic imaging of platelets to count their dense granules has traditionally required electron microscopy, due to the small size of the platelet granules. Dense granules are approximately \SI{150}{\nano\metre} wide and light microscopy has historically been limited to resolutions of  \SIrange{200}{300}{\nano\metre} depending on the technique. However, developments in super-resolution microscopy now allow sufficient resolution in light microscopy to effectively image platelet granules. An automated image analysis pipeline (section~\ref{platelets}) that provides a quantitative, unbiased, rapidly acquired dataset forms the basis for a platelet dense granule disorder diagnostic tool. Using the antigen CD63 as a marker, granules are segmented and assigned to their respective platelets. A proof-of-principle dataset with seven healthy controls and three patients with platelet granule abnormalities caused by HPS, has shown the effectiveness of this technique, and this first dataset is being prepared for publication.

Finally, a workflow is being developed (section~\ref{leukocyteRolling}) for video analysis of leukocyte extravasation. This is part of the innate immune response whereby stages in the leukocyte adhesion cascade from: capture, rolling, slow rolling and firm adhesion, can be imaged under flow in vitro as a time series. Presently, analysis from such videos involves, simply manually counting leukocytes that are slow rolling or firmly adhering. It is proposed that by accurately detecting the leukocytes in each frame and linking them between frames, information about the trajectories of each leukocyte over an endothelial cell surface could be acquired. This analytical technique will allow for a more thorough analysis of the leukocyte adhesion cascade than a manual counting approach.

The development of this suite of image processing tools will help to build a multifaceted picture of the endothelial processes of haemostasis, thrombosis and inflammation. The image processing tools, together with the analysis of their output, allows for the extraction of a great deal of quantitative information to underpin significant advances in the mechanisms that are at the heart of haemostasis and of the initiation of inflammation.

\section{Endothelial Molecular Biology}
Endothelial cells lining the cardiovascular system form the interface between plasma circulating in the blood and the surrounding tissues. At this interface the Weibel-Palade body (WPB), an endothelial specific rapid-response organelle acts as a mediator for vital physiological processes, including inflammation, haemostasis, angiogenesis and vascular tone processes.

The secretory vesicles WPBs contain a set of molecular cargos that can be released in response to endothelial activation from an external stimulus. Activation initiates an exocytotic response to mediate endothelial processes via cellular signalling or mechanical stress.

First discovered in 1964 by Ewald Weibel and George Palade, WPBs were described as a `rod-shaped cytoplasmic component, which consists of a bundle of fine tubules, enveloped by a tightly fitted membrane'~\cite{Weibel1964}. The central role of WPBs in haemostasis is provided by its content of the platelet-recruiting cargo protein von Willebrand factor (vWF)~\cite{Wagner1982}. The multimeric vWF protein underpins primary haemostasis as well as acting to chaperone the blood clotting protein factor VIII. Tubules of vWF are stored within WPBs, giving WPBs a distinctive `cigar-shape', measuring 0.1-0.3~$\mu m$ in diameter and 0.5 to 5.0~$\mu m$ in length~\cite{Ferraro2014}.

Additional components of WPBs have been identified giving further indication as to the organelle's cellular function. Of particular importance are P-selectin~\cite{Bonfanti1989,McEver1989}, Rab27a~\cite{Hannah2003} and CD63~\cite{Vischer1993}. P-selectin is a leukocyte receptor and plays an essential role in the initial recruitment of leukocytes to the site of injury during inflammation. P-selectin is found on the apical surface of endothelial cells and recruits leukocytes when it is exposed to blood via WPB exocytosis, binding the glycoprotein ligand (PSGL)-1 onto the leukocyte surface. The suggested function of Rab27a is to prevent exocytosis of immature WPBs by anchoring them to the actin cytoskeleton~\cite{Nightingale2009}. The tetraspanin CD63 is known to be a component of WPBs, however no endothelial specific function has yet been assigned to this protein. The synthesis and storage of the various WPB protein components is dynamic and can alter rapidly under different cellular conditions.

\subsection{Von Willebrand Factor}
The large multimeric, multi-domain glycoprotein von Willebrand Factor (vWF) is expressed by endothelial cells and platelets. VWF is involved in the initiation of primary and secondary haemostasis~\cite{Metcalf2008}. In primary haemostasis where coagulation has not yet developed, vWF acts to form a loose platelet plug to control bleeding. Following this, secondary haemostasis occurs through an enzymatic coagulation cascade resulting in the formation of cross linked fibrin monomers, stabilising the loose platelet plug to form a clot.

VWF is a cofactor to Factor VIII, preventing its proteolysis within the blood stream~\cite{Sadler1998}. Von Willebrand's disease (vWD) is a condition arising from a quantitative or qualitative abnormalities in vWF. Four hereditary types of vWD exist, the current classification of vWD distinguishes disorders arising from partial type I or complete type III deficiencies and from qualitative defects type II and platelet-vWD or pseudo-vWD. Type II vWD is subdivided into into subtypes A, B, N, and M, each with distinct classes of functional abnormalities~\cite{Ewenstein1997}. Platelet-type vWD or pseudo-vWD is an autosomal dominant genetic defect of the platelets. The von Willebrand factor is qualitatively normal and genetic testing of the von Willebrand gene and vWF protein reveals no mutational alteration. The defect lies in the qualitatively altered glycoprotein1 (GP1) receptor on the platelet membrane which increases its affinity to bind to the von Willebrand factor. 
\subsubsection{VWF in haemostasis}

\subsection{P-selectin}
\subsubsection{The inflammatory cascade}

\section{Microscopy}
\subsection{Immunofluorescence}
\subsection{Phase contrast}

\section{Image Processing}
\subsection{Image enhancement}
\subsection{Segmentation}
\subsection{Morphological processing}
\subsection{Object detection}
